% predict pos vel force. predict N steps ahead so can 1) compensate for delay
% 2) optimize traj. then go to value function.


\documentclass[letterpaper,12pt,fullpage]{article}

\usepackage[left=1in,right=1in,top=1in,bottom=1in]{geometry}
\usepackage{cite}
\usepackage{graphicx}
% \usepackage[dvips]{graphicx}
% \usepackage{epsfig} % for postscript graphics files
  % \graphicspath{{../eps/}}
% \DeclareGraphicsExtensions{.eps}
\usepackage{amsmath}
\usepackage{amssymb}
%\usepackage[cmex10]{amsmath}
%\usepackage{array}
%\usepackage{mdwmath}
%\usepackage{mdwtab}
%\usepackage{eqparbox}
\usepackage[tight,footnotesize]{subfigure}
%\usepackage[caption=false]{caption}
%\usepackage[font=footnotesize]{subfig}
%\usepackage{fixltx2e}
%\usepackage{stfloats}
\usepackage{hyperref}

% correct bad hyphenation here
%\hyphenation{op-tical net-works semi-conduc-tor}

\input latex-commands

\newcommand{\gc}{{\mbox {GravityCompensation}}}
\newcommand{\invdyn}{{\mbox {\tiny ID}}}

\newcommand{\shrinkfig}{\def\baselinestretch{1.0}\small} % 0.9 okay
\newcommand{\shrink}{\def\baselinestretch{1.1}\small} % 0.97 % 0.95 okay

\begin{document}

\title{Estimating Operator Intent\\
(Draft 2.1)}

\author{Alex Ansari, Christopher G. Atkeson, Howie Choset, and Matthew Travers\\
Carnegie Mellon University}

\maketitle

\begin{abstract}
Abstract to be written.
\end{abstract}

\section{Executive Summary}

Executive Summary to be written.

\section{Scope: What is this paper about?}

This paper surveys state of the art technology to recognize
and predict exoskeleton
operator intent.
A companion paper surveys current operator/exoskeleton interfaces~\cite{}.

The focus is on intent estimation that allows a
highly trained and top percentile athletic 
operator to carry a payload that weighs approximately the same amount
as the operator. We envisage these types of exoskeletons to be useful
in carrying protective and safety equipment for SWAT teams, police,
firefighters, and soldiers. 

We focus this survey on intent estimation for lower body tasks (standing, walking,
running, jumping, kicking, dodging, ...).
We do not survey intent estimation for manipulation. 

A future white paper will discuss how intent estimation could be 
made easier with superhuman perception (sensing beyond human capabilities
or modalities).

A future white paper will discuss how intent estimation could be 
made easier with implanted markers and devices, as well as continuous
operator imaging such as ultrasound.

Predictive models of human motion, a major topic of robot learning from\\
observation/demonstration/imitation and computer animation,
will be the subject of a future white
paper~\cite{IEEE06913830,Bagnell}

\section{Introduction: Our Point of View}

{\bf What is intent estimation?}
The further in the future the control system can predict what the operator
is going to do, the easier it is to control the exoskeleton. One way
to think about estimating operator intent is to imagine you are in
a hand-to-hand fight and you want to predict your opponent's next move.
You can also think about playing poker. What ``tells'' does the opponent
have?
You are allowed to instrument your opponent. What sensors would you use?
How would you interpret the sensors?

Here is an example of a robot seeing and moving faster than a human, and
consistantly beating a human opponent in rock-paper-scissors.
\url{https://www.youtube.com/watch?v=Qb5UIPeFClM&feature=player_embedded}

{\bf Individualization of Intent Estimation:}
We recommend each exoskeleton controller
to be used by and optimized for a single operator.
A substantial investment in capturing the operators normal behavior,
operator training and learning, and controller customization should be made.

\section{Intent time scales}

What time scales does intent operate on, and what might we use
to estimate intent on different time scales?

\begin{verbatim}
10 sec - rely on perceiving situation and thinking like your opponent.
         What are the probabilities of various attacks? Responses?
1 sec - measure set/tells to estimate probability of particular behaviors.
100 msec - EMG, muscle state, force helpful here
10 msec - EMG, muscle state, force helpful here and
         immediate sensors: joint pos,vel, IMU, contact forces
1 msec - immediate sensors: joint pos,vel, IMU, contact forces
\end{verbatim}

\section{What signals are there?}

To what extent can we use brain, neural, and muscle electrical signals
(EMG) to anticipate what the operator will do and increase performance?

\begin{verbatim}
- perceive what user sees, hears, feels, smells, tastes.
- superhuman perception (RF signals, UV, IR, ultrasound, ...)
- brain signals
- spinal signals (motorneuron pools)
- motor nerve signals
- muscle electrical signals (EMG)
- muscle force signals (FMG), muscle force at tendons, muscle internal pressure
- sensory nerve signals
- implanted tissue markers, other implanted sensors
- tissue imaging (ultrasound, optical)
- human/exo contact force
- contact strain, deformation rate
- exo joint/IMU sensing
- train user to emit special signals (play a videogame with their body,
  hands, eyes, neural signals, muscle signals, ...)
\end{verbatim}

\section{Signal timing?}

{\it So for inferring the ``intent'' of a human operator of an
exoskeleton,
I would like to put a timeline together of events.
When I say leg EMG or leg muscle I mean
gastrocneumius or vastus lateralis (``fast'' muscles).
What are delays between:
\begin{verbatim}
visual event to leg EMG start
auditory event to leg EMG start
vestibular event to leg EMG start
foot tactile event to leg EMG start
EMG start to 10% leg muscle
EMG start to peak leg muscle
\end{verbatim}
What are your guesses for motorneuron activation to leg EMG activation?
}

{\bf From Hartmut Geyer:}

The traveling time for neural signals is roughly 100m/s in humans.
Thus, to go from any sensory system in the head (vestibular, ocular,
auditory) to the leg muscles will take between 50ms (hip muscles) and
100ms (foot muscles). Once the signal reaches the muscle, it takes
another 5-10ms to be recognized by the surface EMG sensor (electrical
field propagation in muscle tissue is about 0.4m/s).

With that calculation:\\
vis/aud/vestib to EMG: 50-70ms for vastus and about 80-100ms for
gastroc

example: paper on vestibular reflexes reporting 60ms to
90ms:�\url{http://www.ncbi.nlm.nih.gov/pmc/articles/PMC4288134/}
��
If the motor cortex is involved with active calculations, times can
take longer (I don't know numbers for this, but would guess adding
50-100ms). I've heard that in combat soldiers, the motor cortex is
getting by-passed, and they act far more reflexively from sensory
systems directly through the lower brain. So if this is for military
applications, I'd expect to see responses mainly determined by
reflex travel times.

Foot tactile event to gastrocnemius EMG:
\begin{itemize}
\item
traveling time in neural system: 10ms foot sensors to spinal cord +
10ms spinal cord to muscle�
\item
mechanical delay foot to foot sensory cells: 5-10ms
\item
electrical delay muscle activation to EMG: 5-10ms
\item
total: 30-40ms (which is consistent with reported EMG activities
after mechanical disturbance at ankle joint with Bowden cable jerk:
\url{http://jn.physiology.org/content/76/2/1112.short�reports 42ms)}
\end{itemize}

Foot tactile event to vastus EMG:\\�
- same as gastrocnemius less about 5ms for shorter travel time down
the spinal cord to the muscle.

EMG start to 10\% leg muscle force:\\
- this gets tricky: EMG has fixed delay from muscle activation due to
travel of electrical field to muscle surface (5-10ms, see top). Delay
of muscle force production follows low pass filter function (calcium
ion dynamics), often modeled as single pole filter with 10ms
characteristic time. At 10\% force, both EMG and leg force could
actually occur at the same time.�

EMG start to peak muscle force:\\
typically, (full) force production trails EMG by about 30ms in fast
muscles (electromechanical delay, due to calcium ion dynamics
traveling from sarkoplasmatic reticulum to cross bridge binding
sites, \url{http://www.ncbi.nlm.nih.gov/pubmed/527577})

alpha motoneuron activation to leg EMG:\\
- travel time spinal cord to muscle (5ms for vastus, 10ms for
gastocnemius) plus 5-10ms before muscle activation gets recognized in
surface EMG:\\
net vastus: 10-15ms\\
net gastrocnemius: 15-20ms

\section{What can be measured or estimated?}

{\bf EMG:}
Electrical activity was measured using surface electrodes,
high-pass filtered at 20 Hz, rectified, low-pass filtered at 6 Hz,
offset by a small value of -0.008, and amplified by a gain of 283
to obtain the desired torque.~\cite{IEEE07139980}

07222568.pdf recognition

Electromyography (EMG)
signals are extremely noisy and it is not easy to extract a
meaningful data form the raw signal. In addition, it takes a long
time to attach the EMG electrodes to the body and calibrate the
sensors.
\begin{verbatim}
T. Hayashi, H. Kawamoto, and Y. Sankai, "Control method of robot
suit HAL working as operator's muscle using biological and dynamical
information," in Intelligent Robots and Systems, 2005.(IROS 2005).
2005 IEEE/RSJ International Conference on, 2005, pp. 3063-3068.
H. Kawamoto and Y. Sankai, "Power assist system HAL-3 for gait
disorder person," in Computers helping people with special needs, ed:
Springer, 2002, pp. 196-203.
\end{verbatim}

Muscle movement leads to surface electrodes losing good contact, fully detaching, or
even ripping off (especially for large fast
movements, the worst time).
Perspiration changes measurement. EMG measurement
needs to be calibrated for each electrode application and periodically during use.

FMG-[06913842]

{\bf Foot contact and force/torque sensing:}
\begin{verbatim}
[8] K. Kong and M. Tomizuka, “A Gait Monitoring System Based on Air
Pressure Sensors Embedded in a Shoe,” IEEE/ASME Transactions on
Mechatronics, vol. 14, pp. 358–370, 2009.
\end{verbatim}

\section{How can the operator communicate with the exoskeleton}

\begin{verbatim}
Operator uses high frequency range or patterns of output to
signal what to do
 - selection
 - parameterization (how far, how high, how fast)
\end{verbatim}

The operator controls more than the pose and motion of the
exoskeleton. What if there are "pointing" sensors or communication
devices that need to be aimed at or track an area of interest? What if
there are additional (physical or virtual) pan/tilt/zoom cameras
pointing to the side and rear or line of sight secure communications,
for example. What about other controls, such as power assist level or
thermal control? How can an operator naturally express intent and
control these additional degress of freedom?

\section{Operator load and fatigue}

To what extent can task recognition and "autonomy" reduce operator
load and fatigue?

\section{Discussion}

Discussion to be written.

\section{Conclusions and Recommendations}

1) We recommend using a Bayesian (probabilistic) framework for intent recognition
and prediction. This is similar to using an extended
Kalman filter and predictor for intent.
A classifying filter that tracks probabilities of different behavior selections
and their parameter selections can be implemented with a multiple model Kalman
filter or a particle filter.

2) We recommend developing an intent prediction system that can use
as many sensors and measurements of the operator as are possible within cost limits,
including at least external computer vision and audition, exoskeleton-world contact
force,
EMG, muscle pressure, force between the operator and the
exoskeleton,
speech acquisition, keyboards built into the exoskeleton handles or gloves,
and accelerometers/microphones/vibration-sensors attached to the operator skin.
This will enable the implementation of a variety of intent prediction approaches,
and avoid ``putting all our eggs in one basket.''

3) We recommend each exoskeleton controller
to be used by and optimized for a single operator.
A substantial investment in capturing the operators normal behavior,
operator training and learning, and controller customization should be made.

More Conclusions and Recommendations to be written.

\bibliographystyle{plain}
\bibliography{exo}

\end{document}


