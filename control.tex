\documentclass[letterpaper,12pt,fullpage]{article}

\usepackage[left=1in,right=1in,top=1in,bottom=1in]{geometry}
\usepackage{cite}
\usepackage{graphicx}
% \usepackage[dvips]{graphicx}
% \usepackage{epsfig} % for postscript graphics files
  % \graphicspath{{../eps/}}
% \DeclareGraphicsExtensions{.eps}
\usepackage{amsmath}
\usepackage{amssymb}
%\usepackage[cmex10]{amsmath}
%\usepackage{array}
%\usepackage{mdwmath}
%\usepackage{mdwtab}
%\usepackage{eqparbox}
\usepackage[tight,footnotesize]{subfigure}
%\usepackage[caption=false]{caption}
%\usepackage[font=footnotesize]{subfig}
%\usepackage{fixltx2e}
%\usepackage{stfloats}
\usepackage{hyperref}

% correct bad hyphenation here
%\hyphenation{op-tical net-works semi-conduc-tor}

\input latex-commands

\newcommand{\gc}{{\mbox {GravityCompensation}}}
\newcommand{\invdyn}{{\mbox {\tiny ID}}}

\newcommand{\shrinkfig}{\def\baselinestretch{1.0}\small} % 0.9 okay
\newcommand{\shrink}{\def\baselinestretch{1.1}\small} % 0.97 % 0.95 okay

\begin{document}

\title{A Survey of Possible Exoskeleton Control Architectures and
Algorithms\\
(Draft 2.1)}

\author{Alex Ansari, Christopher G. Atkeson, Howie Choset, and Matthew Travers\\
Carnegie Mellon University}

\maketitle

\begin{abstract}
Abstract to be written.
\end{abstract}

\section{Executive Summary}

Executive Summary to be written.

\section{Scope: What is this paper about?}

This paper surveys possible exoskeleton control architectures and
algorithms.
One goal is to address the question:
{\it How do exoskeleton control approaches compare with 
approaches to humanoid robot control?}
A companion paper surveys implemented exoskeleton control architectures
and algorithms~\cite{}.

The focus is on exoskeleton control that allows a
highly trained and top percentile athletic 
operator to carry a payload that weighs approximately the same amount
as the operator. We envisage these types of exoskeletons to be useful
in carrying protective and safety equipment for SWAT teams, police,
firefighters, and soldiers. 

We expect each exoskeleton controller
to be used by and optimized for a single operator.
A substantial investment in capturing the operators normal behavior,
operator training and learning, and controller customization can be made.

We focus this survey on exoskeleton control for lower body tasks (standing, walking,
running, jumping, kicking, dodging, ...).
We do not survey exoskeleton control for human arms or manipulation. 

We treat the torso and helmet of the exoskeleton as the payload,
and focus on a lower body exoskeleton to support any payloads that
are on the torso or head.

%\section{Symbiosis and Autonomy, Handling Errors, and Superhuman Reflexes}
A later white paper will discuss
possible combinations of response to user and exoskeleton
autonomy including balance control; response to trips, slips,
stumbles, and fumbles; task assistance and guidance (guide operator to
doorknob, button, or light switch); superhuman response to external
perturbations such as projectiles and explosions; and autonomous
execution (operator/exoskeleton symbiosis with multi-tasking)?

\subsection{Let's keep it simple}

Terms like impedance and admittance control are used, but are often confusing,
as in controller design one can choose from several possible
inputs into the exoskeleton (exoskeleton positions,
velocities, accelerations, operator-exo contact forces and
exo-world contact forces), and choose from several possible exoskeleton
outputs: exoskeleton actuator forces and torques, exoskeleton
motion (position, velocity, and acceleration), as well as operator-exo
contact forces or exo-world contact forces.

A useful background paper: \url{http://summerschool.stiff-project.org/fileadmin/pdf/1804_C19.pdf}

Variants of nonlinear feedback control such as feedback linearization or
sliding mode control are largely ignored in this paper.
Once the decision to use feedback control
based on a set of observable quantities and with particular outputs is made,
one can try out the various linear and nonlinear feedback control paradigms
to see what works well.

Variants of function approximation methods such as lookup tables, fuzzy logic,
sigmoidal neural networks, radial basis functions, and locally weighted regression
are also largely ignored. Once the decision to use a function approximator
and what the inputs and outputs are has been made, one can try out the various
approaches to see what works well.

Same for optimization methods.

Same for constraint enforcement (such as avoiding self-collisions)
in either optimization or feedback control (barrier
Lyapunov functions (BLF)~\cite{IEEE06911561}).

Stability proofs of any of these methods should be viewed skeptically due
to the unmodeled operator-exoskeleton and exoskeleton-world contact dynamics, 
actuator unmodeled dynamics, joint play and exoskeleton
structural deformation, and controller
time delays which are typically ignored, especially in proofs involving
passivity arguments or Lyapunov functions.
We ignore stability proofs as well.

\section{What are some design philosophy alternatives?}

\subsection{``Invisible'' exoskeleton}

Can we build an exoskeleton allows the operator to behave naturally
and exerts little or no force on the operator?

This could be achieved through active control using feedback of the
operator-exoskeleton forces. Additional gravity, friction, actuator, and
inertial compensation (inverse dynamics) can improve performance.

\subsection{``Natural'' exoskeleton}

Can we build an exoskeleton allows the operator to behave naturally?
The operator can feel the exoskeleton, but can overpower it when necessary.

Bypass valves and clutches could allow the operator to physically ``take over''
and push the relatively lightweight leg mounted exoskeleton around.

``Passive'' degrees of freedom can have physical or ``virtual'' springs
applied to assist in support (all abduct/adduct DOF, rotational DOF).
These aids can be clutched out or turned off when necessary.
``Virtual'' springs (springs implemented actively through exoskeleton
actuators) can be modulated.

\subsection{``Symbiotic'' exoskeleton}

Can we build an exoskeleton that allows the operator/exoskeleton team
to be effective, but not necessarily behave naturally? 
The operator needs to learn to ``fly'' the suit,
just as operators learn to operate parachutes,
wing suits, diving equipment, high and low temperature protective
suits, firefighting equipment, and high altitude flight
and parachute suits.

The exoskeleton can adapt its behavior to the task.
For example, for jumping, the exoskeleton can operate like a pogo 
stick \url{https://www.youtube.com/watch?v=lbp41vWP4o4}, 
and for running it could act like jumping 
stilts \url{https://www.youtube.com/watch?v=9ZOd7yEyhwI},

\section{Effects of underlying exoskeleton actuator technology}

One basic distinction is whether the actuators are thought of as position
or force sources. Often this is revealed by the design of the ``low level''
controller. Note that this is often somewhat confusing. Electric motors
are torque sources (low impedance), but when a high gear ratio transmission
is added (such as a harmonic drive) the whole system becomes high impedance
and thus is best thought of as a position (or velocity) source. Therefore,
position control is performed by the low level controller. Hydraulic actuators
are high impedance, but when a force or load sensor is added (or piston differential
oil pressure is used as a force sensor), the whole system becomes low impedance
and is best thought of as a force source. Force (or joint torque) control is
performed by the low level controller. Force or torque can be added to an
electromechanical drive train (electric motor plus gears or motor plus ballscrew)
to make it lower impedance and more like a force source as well.

\section{Recommended Control Architecture}

We recommend the following hierarchical control architecture~\cite{IEEE06907051}:

{\bf Low Level Control:} Present interface to actuators and
high bandwidth feedback control, degrees of freedom,
or synergies of either:
\begin{itemize}
\item
Position control.
\item
Pure force or torque control.
\item
Impedance control: specify desired position (for spring), and generate
stiffness, damping, and potentially a modified inertia or moment of inertia.
Modifying inertia is more dangerous than providing active stiffness
or damping.
Or (almost equivalently),
specify a desired force and change the force according to position, velocity,
and acceleration.
\end{itemize} 
One important synergy is coordinated control of the center of mass location and
velocity for balance.
PID or fancier feedback control laws are implemented at this level.
Gravity, friction, actuator dynamics, and other feedforward disturbance
compensation are implemented at this level, as well as any inverse dynamics
or other types of feedforward reference tracking control.
Disturbance observers for model errors and contact/wind forces
are implemented at this level.
Typically this is done by a high sampling rate global servo~\cite{Atlas-robot}
or distibuted limb or joint-level high sampling rate servos~\cite{Sarcos-robot}.
Often, this controller is provided by the manufacturer of the robot
or exoskeleton and runs on a separate computer system from the
``user's'' or task/application controller~\cite{Sarcos,Atlas}.

{\bf ``Phoneme'' Control:} Mid-level controllers
for relatively constant phases of behavior.
Often involves position or velocity targets (goto or
stay-at X), trajectory references, ...
Example from walking: left-stance, double-support-1, right-stance, double-support-2, ...;
Example from running: left-foot-down, flight-1, right-foot-down, flight-2, ...;
Example from horizontal jumping: hjump-pushoff, hjump-flight, hjump-prepare-landing, hjump-impact, hjump-balance, ...;
Finer resolution may be needed. For example a walking stance period may
be broken up into heel-strike, first-half, second-half, pushoff, and toe-off.
The first and second halves of stance are divided by when the center of mass
passes over the ankle.
Local inverse kinematics (finding small joint motions for small target changes)
is performed at this level.
This level often provides the behavioral primitives or ``verbs'' for robot programming.

{\bf ``Word'' Control:} Mid-level controllers for sequences of ``phoneme'' behavioral
units.
Often involves finite state machines, timers, if-then condition testing, gain
scheduling, ...
Examples: walk-to-location( x=7.8m, y=2.3m ), run-to-location( speed=3.7m/s ),
horizontal-jump( height= 2m, length= 5m ).
Global inverse kinematics (selection of solution branch) is performed at this level.
This level often determines the control flow of a behavior.

{\bf Highest Level Control} aka {\bf Behavior Selection:} 
Answer the question: ``What do I do next?''
Select behaviors and behavioral parameters (targets,
speeds, durations).
Often involves finite state machines. timers, if-then condition testing, ...
This level often involves a great deal of human operator online interaction,
or is completely done by the operator.

{\bf Error Handling:} Errors must be detected and handled by parameter (target)
adjustment and/or behavior switching
at all of these levels.

This control architecture is intuitive, abstracts the hardware 
and lower level control in a straightforward way, is easy to program, and is
commonly used. Variants of this architecture were used by almost all teams
in the DARPA Robotics Challenge~\cite{}.

Closely related control architectures combine Phoneme and Word control,
or combine Phoneme, Word, and Behavior Selection.
This is not a major design change.

An alternative behavioral architecture might not have explicit levels but instead
has a ``soup'' of behaviors that compete or cooperate for the ability to
control the robot or exoskeleton. Subsumption architecture and other behavior-based
architectures are closer to the ``soup'' approach and to some extent
use inhibition and excitation
among behaviors to select active behaviors.

Another alternative behavioral architecture does not have explicit separate behaviors,
but one or a small number of functions or policies. 
We are seeing more of these architectures
due to the success and popularity of function approximation such as
deep learning.

\section{Exoskeleton dynamics}

Exoskeletons are similar to robots, but there are several important differences.
There are two sets of contact forces: operator-exoskeleton (which we denote
with a subscript $o$ as in $\vf_o$) and exoskeleton-world (which we denote
with a subscript $w$ as in $\vf_w$). If the operator is tightly strapped or
rigidly held by other types of physical constraints, we can lump the operator
body parts with the exoskelton parts which they are attached to in terms
of link inertias, location of center of mass, and moments of inertia,
and just think of the whole system as one robot.
We can also separate out the operator dynamics if this is more convenient
(we separate them out and ignore them in what follows).
If the operator is loosely connected to the exoskeleton, this introduces a
huge modeling and control problem, and would also be very dangerous for the
operator (as riding a bull or bucking horse in a rodeo is dangerous to a rider).

The dynamics of a revolute exoskeleton are:
\begin{equation}
\mM(\vq) \ddot{\vq} + \mC(\vq,\dot{\vq}) + \mG(\vq) + \mF(\vq,\dot{\vq})
+ \mJ_w^{\tr}(\vq) \vf_w = \vtau + \mJ_o^{\tr}(\vq) \vf_o
\label{eq:dynamics}
\end{equation}
where $\vq$ are the joint angles and position of the ``root'' of
the exoskeleton, $\dot{\vq}$ and $\ddot{\vq}$ are the corresponding
velocities and accelerations, $\mM$ is the inertia matrix, $\mC()$ are the Coriolis and
centripetal forces, $\mG()$ are the gravitational forces, $\mF()$ are the
friction forces, $\mJ_w^{\tr}(\vq) \vf_w$ (Jacobian matrix multiplied with the
contact forces)
are the exoskeleton-world
contact forces 
expressed as exoskeleton joint torques, 
$\vtau$ are the exoskeleton
joint torques,
and $\mJ_o^{\tr}(\vq) \vf_o$
are the operator-exoskelton forces expressed
as exoskeleton joint torques.
$\vtau$ is augmented with zeros corresponding to the dimensions of $\dot{\vq}$
that correspond to root velocities, since those dimensions are not actuated.

To understand these equations better, it is useful to have all Jacobians equal
to the identity matrix. This means the operator is directly applying torques
at the exoskeleton joints ($\vtau_o$) (a realistic assumption), 
and so is the world ($\vtau_w$) (very unrealistic assumption). In this case:
\begin{equation}
\mM(\vq) \ddot{\vq} + \mC(\vq,\dot{\vq}) + \mG(\vq) + \mF(\vq,\dot{\vq})
+ \vtau_w = \vtau + \vtau_o
\end{equation}

Note that we are ignoring the operator dynamics and operator muscle and tissue
forces resulting in operator joint torques. We will add these in later (or maybe
we won't ???).

We are also ignoring actuator dynamics, which we can handle by augmenting 
the state vector.

\section{Passive control with no active control}

One control option is to use only passive controls, and no active controls
or actuators.
Such a system would rely on counterbalancing, mechanical springs, air springs, etc.
to reduce forces. 
An operator would rely on mechanical backdrivability to drive
exokeleton to do tasks.
In this case, 
$\vtau$ is generated by passive devices such as springs or dampers.

\section{Open loop control}

Another control option is open loop control, in which actuator commands are
constant, or a function of time or phase of a behavior or task
(at least on the small time scale).
Commands or behaviors may be selected at a relatively coarse time interval
by the operator or by the exoskeleton controller.
This approach typically 
uses position and velocity control in either joint or
Cartesian coordinates
with either pre-generated references, selected targets,
or online generated trajectories to move to targets or perform tasks.
The operator relies on mechanical backdrivability 
to correct exoskeleton behavior.
In this case, it is common to generate the exoskeleton actuator commands using
a time or phase index:
$\vtau = \vtau_{ff}(t)$ where $\vtau_{ff}$ is a time
dependent feedforward torque vector.

Recognizing/estimating phase goes here.
Phase Triggered References could go here.
\begin{verbatim}
H.Kawamoto and Y.Sankai, “Power Assist Method Based on Phase
Sequence Driven by Interaction between Human and Robot Suit,”
Proceedings of the IEEE International Workshop on Robot and Human
Interactive Communication, pp. 491–496, 2004.
\end{verbatim}

Motion capture, motion libraries goes here.

Often finite state machines whose states correspond to left stance, double support 1,
right stance, and double support 2 are used to select behaviors. Gait transitions
can be triggered by foot touchdown and liftoff events, or
displacement of the Center of Mass (CoM).

Central pattern generator

Iterative learning control to acquire $\vtau_{ff}(t)$ for desired trajectories.

\section{Active gravity compensation}

Active gravity compensation is similar 
to counterbalancing or physical gravity compensation, except it is done
with exoskeleton actuators to cancel the exoskeleton weight (but not inertia).
\begin{equation}
\vtau = \hat{\mG}(\hat{\vq})
\end{equation}
where $\vtau$ are the exoskeleton joint torques, 
and $\hat{\vq}$ is the estimated configuration of the exoskeleton,
The operator relies on mechanical backdrivability to drive
the exokeleton to do tasks.

Sarcos Primus Humanoid (ATR) video of active gravity compenstation applied
to a humanoid robot:\\
\url{https://www.youtube.com/watch?v=KNxxLm4sPys}

A simple approximation for a lower body exoskeleton with a heavy payload at the
torso is:
\begin{equation}
\vtau \approx \hat{\mJ}^{\tr}(\hat{\vq}) (\hat{m} \hat{\vg})
\end{equation}
where $\hat{\mJ}$ is the estimated Jacobian matrix,
$\hat{m}$ is the estimated
weight of the exoskeleton (which could include the operator),
and $\hat{\vg}$ is the gravity vector (it is estimated due to possible orientation
errors in the controller's estimate of vertical). 

\section{Friction compensation}

Friction can also be compensated.
\begin{equation}
\vtau = \hat{\mF}(\hat{\vq},\hat{\dot{\vq}}) 
\end{equation}
are the estimated friction torques at the estimated exoskeleton configuration
and velocity.
Note that these may be load dependent, and may require additional state to
correctly handle stiction and hysteresis effects.

\section{Online inverse dynamics control}

The goal here is to map from desired motion (in this case desired
accelerations) to exoskeleton actuator commands (and desired operator
and world contact forces).
The exoskeleton actuator torques
attempt to match the torques caused by inertial, Coriolis, centripetal,
gravitational, frictional, and exoskeleton-world contact forces.
\begin{equation}
\vtau = \vtau_{\invdyn} 
= \hat{\mM}(\hat{\vq}) \ddot{\vq}_d
+ \hat{\mC}(\hat{\vq},\hat{\dot{\vq}})
+ \hat{\mG}(\hat{\vq})
+ \hat{\mF}(\hat{\vq},\hat{\dot{\vq}})
+ \hat{\mJ}_w^{\tr}(\hat{\vq}) \hat{\vf}_w
\end{equation}
If there are actuator dynamics or other dynamics, we can include those
dynamics as well.

These force/torque estimates rely on estimated dynamic models and their parameters.
If the models and estimated state (position and velocity) are perfect, we get
perfect cancellation of most of the dynamics:
\begin{equation}
\mM(\vq) (\ddot{\vq} - \ddot{\vq}_d) = \mJ_o^{\tr}(\vq) \vf_o
\end{equation}
If $\ddot{\vq}_d$ is set to zero, the operator directly drives the acceleration
of the exoskeleton with an effective inertia of $(\mM^{-1}(\vq) \mJ_o^{\tr}(\vq))^{-1}$
\begin{equation}
\ddot{\vq} = \mM^{-1}(\vq) \mJ_o^{\tr}(\vq) \vf_o
\label{eq:cancel}
\end{equation}
$\ddot{\vq}_d$ can be used to drive the exoskeleton along trajectories, and
the operator can directly modify the trajectory.

Simplifying equation~\ref{eq:cancel} by 
assuming the operator directly applies torques at the
exoskeleton joints we get:
\begin{equation}
\mM (\vq) \ddot{\vq} = \vtau_o
\end{equation}
The operator sees the true inertia of the exoskeleton, but no gravitational
loads and no other dynamics (Coriolis, centripetal, or frictional).

Note that we cannot change the apparent inertia of the exoskeleton without
some form of acceleration or force feedback.
With acceleration feedback we can make the exoskeleton have a different inertia.
Setting $\ddot{\vq}_d$ to zero and adding acceleration feedback:
\begin{equation}
\vtau = \vtau_{\invdyn} - (\mM_d(\hat{\vq}) - \hat{\mM}(\hat{\vq})) \hat{\ddot{\vq}}
\label{eq:change-inertia1}
\end{equation}
sets the apparent inertia to $\mM_d$ if the dynamic models are perfect, and
the state estimation produces not only perfect positions and velocities but
also perfect acceleration estimates. In the case 
where the operator directly applies torques at the
exoskeleton joints we get:
\begin{equation}
\mM_d (\vq) \ddot{\vq} = \vtau_o
\end{equation}
In robotics this is usually considered a bad idea if $\mM_d$ is smaller (a complex
issue for a matrix) than $\mM$ at any $\vq$. This type of control is vulnerable
to unmodeled dynamics. However, in an exoskeleton, since we have not tried to
cancel the human's dynamics, the human stabilizes a possible unstable exoskeleton
(if the straps are tight enough).

With force feedback we can also make the exoskeleton have a different inertia.
Setting $\ddot{\vq}_d$ to zero and adding force feedback:
\begin{equation}
\vtau = \vtau_{\invdyn} + \mK \hat{\vf}_o
\end{equation}
In the case where the force measurements are perfect and 
the operator directly applies torques at the
exoskeleton joints we get:
\begin{equation}
(\mK + \mI)^{-1} \mM (\vq) \ddot{\vq} = \vtau_o
\end{equation}
By setting $\mK = \mM^{-1}_d (\vq) \mM (\vq) - \mI$ we can achieve
\begin{equation}
\mM_d (\vq) \ddot{\vq} = \vtau_o
\end{equation}

Online inverse dynamics control is a generalization of 
active gravity compensation to include
inertial, Coriolis, and centripetal forces (forces due to acceleration and
velocity), and potentially frictional forces.
Online inverse dynamics can be added to open loop behavior execution
and many other types of control such as operator force
feedback control, impedance control, and admittance control.
Computed torque control and feedback linearization are forms of online
inverse dynamics.

A key question is where does desired acceleration come from?
\begin{verbatim}
  Operator force control.
  Receding Horizon Control of optimal trajectory of 
    idealized LIPM+flywheel model (Atkeson)
  intent prediction (see above)
  neuromuscular model (Geyer, Wang, Van de Panne)
  hybrid: Optimal control + neuromuscular model (Atkeson NSF Proposal)
    - NMM provides bias through changes to optimization criterion
    - NMM provides hard constraints
    - NMM replaces components of model-based optimization scheme
\end{verbatim}

\section{Operator-Exoskeleton Impedance Control}

What is impedance?
\url{https://en.wikipedia.org/wiki/Mechanical_impedance}
Stiffness, damping, and mass are components of an impedance, as
they map kinematic variables (position, velocity, and acceleration)
to forces and torques.

The goal here is to make the exoskeleton imitate a desired linear dynamic system
from the point of view of the operator.
A future white paper will discuss how to make the exoskeleton 
imitate a desired linear dynamic system 
from the point of view of external perturbations.
In the case where there is no force sensing between the operator and the
exoskeleton, the exoskeleton controller maps from exoskeleton configuration
and velocity to desired joint/actuator/synergy forces.
In this case the actuators need to generate forces and torques rather than
positions or angles, or linear or angular velocities.

Here we make the exoskeleton appear as a locally linear impedance in joint
coordinates to the operator.
We can also make the exoskeleton appear as a locally linear impedance to the
operator in some other coordinate system, such as Cartesian coordinates.
A different goal is to make the exoskeleton appear as a locally linear impedance to
the outside world in some coordinate system.

To make the exoskeleton appear as a locally linear impedance in joint
coordinates to the operator, we add position and velocity feedback to $\vtau$:
\begin{equation}
\vtau = \vtau_{\invdyn} - \mK_{\vq} ( \hat{\vq} - \vq_d ) - \mK_{\dot{\vq}} \hat{\dot{\vq}}
\label{eq:impedance}
\end{equation}
so if the dynamic models and state estimation are perfect we get this impedance:
\begin{equation}
\mM \ddot{\vq} - \mK_{\vq} ( \vq - \vq_d ) - \mK_{\dot{\vq}} \dot{\vq} = \vtau_o
\end{equation}
and the operator sees the desirecd impedance.

Although impedance control is extensively utilized in rehabilitation
robotics [18], limited number of studies has been focused on
this method for power augmentation [19, 20].
\begin{verbatim}
[18] R. Riener, L. Lunenburger, S. Jezernik, M. Anderschitz, G. Colombo,
and V. Dietz, "Patient-cooperative strategies for robot-aided treadmill
training: first experimental results," Neural Systems and Rehabilitation
Engineering, IEEE Transactions on, vol. 13, pp. 380-394, 2005.
[19] B.-K. Lee, H.-D. Lee, J.-y. Lee, K. Shin, J.-S. Han, and C.-S. Han,
"Development of dynamic model-based controller for upper limb
exoskeleton robot," in Robotics and Automation (ICRA), 2012 IEEE
International Conference on, 2012, pp. 3173-3178.
[20] W. Yu, J. Rosen, and X. Li, "PID admittance control for an upper limb
exoskeleton," in American Control Conference (ACC), 2011, 2011, pp.
1124-1129.
\end{verbatim}

\section{Force feedback and get out of the way control (inverse dynamics version)}

The goal here is to use force sensing between the operator and the
exoskeleton to ultimately generate exoskeleton velocities or angular velocities,
imitating a desired linear dynamic system.
This type of control is often referred to as admittance control as the
exoskeleton maps contact forces into joint motion.
Inverse dynamics control
can be used to improve this type of control performance.

What is admittance?
\url{https://en.wikipedia.org/wiki/Admittance}
Compliance, inverse damping, inverse mass are components of an admittance,
as they map force and torques to kinematic variables (position, velocity, and
acceleration). 

We can implement force control 
if measurements of the contact forces with
the exoskeleton are available.
The exoskeleton moves as to create as little force as possible 
assuming the actuators are torque sources.
\begin{equation}
\vtau = \vtau_{\invdyn} - \mK_{\vf} \mJ^{\tr} ( \hat{\vf} - \vf_d )
\end{equation}
$\mK_{\vf}$ is a gain matrix, $\mJ$ is the appropriate Jacobian matrix for
force $\vf$
Note that this does not require inverting the Jacobian matrix.
We will see that the position control version does invert the Jacobian matrix.

Applying this to force between the operator and the exoskeleton, and assuming
that the operator directly applies exoskeleton joint torques,
\begin{equation}
\vtau = \vtau_{\invdyn} - \mK_{\vtau_o} ( \hat{\vtau}_o - {\vtau_o}_d )
\end{equation}
So if our dynamic models and state estimation are perfect, we get:
\begin{equation}
\ddot{\vq} = - \mM^{-1} (\vq) \mK_{\vtau_o} ( \hat{\vtau}_o - {\vtau_o}_d )
\end{equation}

One can add force damping terms, which sometimes help:
\begin{equation}
\vtau = \vtau_{\invdyn} - \mK_{\vtau_o} ( \hat{\vtau}_o - {\vtau_o}_d )
- \mK_{\dot{\vtau}_o} \hat{\dot{\vtau}}_o
\end{equation}

Note that impedance control can also be implemented using force control
by setting:
\begin{equation}
{\vtau_o}_d = \mK_{\vq}( \hat{\vq} - \vq_d ) + \mK_{\dot{\vq}} \hat{\dot{\vq}}
\end{equation}
in addition to directly including the stiffness and damping terms in
equation~\ref{eq:impedance}. 

\section{Mapping External Forces To Operator Forces Using Force Control}

So far we have ignored or canceled the effect of 
the exoskeleton contact with the outside world.
A different control objective could be to map the external forces to 
desired operator forces:
\begin{equation}
{\vf_o}_d = F_{ow} ( \vf_w )
\end{equation}
and cancel the external forces not transferred in the inverse dynamics.
Everything in the above equation is estimated, so we are dropping the hats 
$\hat{\mbox{}}$.

For example, we could want the operator to feel a scaled (scale factor $\alpha$) 
version of the external
forces, as if they were contacting the operator directly.
\begin{equation}
{\vf_o}_d = \alpha J_o^{-\tr}(\vq) J_w^{\tr}(\vq) \vf_w
\end{equation}
and we scale the $\mJ_w^{\tr}(\vq) \vf_w$ term in the inverse dynamics
by $(1-\alpha)$
We are assuming we have any necessary force sensing between the exoskeleton and
the world.

If the operator applies joint torques, this simplifies to
\begin{equation}
{\vtau_o}_d = \alpha J_w^{\tr}(\vq) ( \vf_w )
\end{equation}
and the inversion of a Jacobian matrix is unnecessary.

This would provide a way of enabling the operator to perceive and control
external contact forces.

\section{Autonomous exoskeleton control}

\subsection{Autonomous exoskeleton position/velocity control}

A control objective could be to have the exoskeleton to move autonomously,
which is useful if the operator would like to rest, is attending to some other
task, or is disabled.

This is usually done by using stored trajectories or trajectories generated
online that specify desired position $\vq_d$, 
desired velocity $\dot{\vq}_d$, and
desired acceleration $\ddot{\vq}_d$.
The feedforward method is to compute the needed torques (potentially offline)
using inverse
dynamics (including the operator link inertias, link centers of mass, and link
moments of inertia in the dynamic model) based on only the desired postion, 
desired velocity, and desired acceleration. Trajectory tracking errors are compensated
for by feedback control designed using some other method. In most robotics
fixed gain independent joint linear control is used.
This method is safest in the sense that the feedback controller can be tested
and verified independently of the desired trajectories and dynamic models.

Computed torque approaches use online inverse dynamics computed from the estimated
position, estimated velocity, and desired acceleration.
In this case the feedback control to compensate for trajectory tracking errors can
be independent of the inverse dynamics and its modeling errors, or the desired
acceleration can be modified with feedback terms.
\begin{equation}
\ddot{\vq}_d^* = \ddot{\vq}_{d} + \mK_{\vq}^{\ddot{\vq}} ( \hat{\vq} - \vq_d ) 
+ \mK_{\dot{\vq}}^{\ddot{\vq}} ( \hat{\dot{\vq}} - \dot{\vq_d} )
\end{equation}
The problem with putting the inverse dynamics in the feedback loop is that the
effective gains of this control vary widely with the configuration $\vq$.
Unmodeled dynamics (that also vary with $\vq$) limit the size of the gain matrices
$\mK_{\vq}^{\ddot{\vq}}$ and $\mK_{\dot{\vq}}^{\ddot{\vq}}$. Because so many quantities
are varying quite a bit, constant gain matrices may be quite low for a worst case
design. It is also very difficult to verify this type of nonlinear feedback controller.
For this reason most current approaches to humanoid
robots design the feedback controller
separately and use constant gain independent joint linear feedback control.

\subsection{Autonomous exoskeleton-world force control}

A control objective could be to have the exoskeleton to apply a desired force.

To be written.

\begin{verbatim}
include J' term
\end{verbatim}

Why isn't this identical to operator force control?

\subsection{Autonomous exoskeleton-world impedance control}

Another control objective would be to allow the exoskeleton-world contact to
be compliant or have an assigned impedance.

To be written.

\begin{verbatim}
Cartesis version - j-1
joint version, w jKj =k
\end{verbatim}

Why isn't this identical to operator impedance control?

\section{Using Online Optimization: Quadratic Programming}

The dominant paradigm in force controlled (typically
hydraulic) full size humanoid robots is to handle constraints in inverse dynamics
such as torque and center of pressure limits using online quadratic programming
to solve the inverse dynamics equations. This approach can also be used
to implement and trade off among the many control objectives described so far.

Quadratic programming~\cite{Wikipedia} solves the following problem:
Minimize
\begin{equation}
\vx^{\tr} \mQ \vx + c^{\tr} \vx
\end{equation}
with inequality constraints
\begin{equation}
\mA_1 \vx \le \vb_1
\end{equation}
and equality constraints
\begin{equation}
\mA_2 \vx = \vb_2
\end{equation}
The optimization variable $\vx$ includes accelerations $\ddot{\vq}$, joint
torques $\vtau$, and contact forces $\vf$.
For robots the discrepancy between actual and desired accelerations, joint
torques, and contact forces are all minimized:
\begin{equation}
C_r(\vx) = (\vq - \vq_d)^{\tr} \mW_{\vq} (\vq - \vq_d) + \vtau^{\tr} \mW_{\vtau} \vtau
+ \vf^{\tr} \mW_{\vf} \vf
\end{equation}
The dynamics (equation~\ref{eq:dynamics}) are enforced as an equality constraint.
Torque limits, friction limits, center of pressure location constraints,
and unidirectional contact constraints are all
enforced as inequality constraints.
Variants of this optimization
were performed every 1-2ms by various teams to control the
Atlas robots in the DARPA Robotics Challenge.

This approach can be used to achieve desired joint or synergy (such as center
of mass) accelerations, center of mass dynamic behavior, center of pressure location,
weight distribution, maintenance of a desired body part position, orientation,
velocity, or trajectory. and joint/actuator position and velocity limits (indirectly).

In the case of exoskeletons, part of the optimization criterion could be to
minimize contact forces on the operator:
\begin{equation}
C_e(\vx) = C_r(\vx) + \vf^{\tr}_o \mW_{\vf_o} \vf_o
\end{equation}
Operator force feedback and impedance control could be implemented by specifying a
desired operator force ${\vf_o}_d$:
\begin{equation}
C_e(\vx) = C_r(\vx) + (\vf_o - {\vf_o}_d)^{\tr} \mW_{\vf_o} (\vf_o - {\vf_o}_d)
\end{equation}

To implement operator impedance control we set the desired operator contact force to:
\begin{equation}
{\vf_o}_d = \mJ_o^{-1} ( \mK_{\vq}( \hat{\vq} - \vq_d ) + \mK_{\dot{\vq}} \hat{\dot{\vq}} )
\end{equation}
Assuming that the operator directly applies exoskeleton joint torques,
\begin{equation}
C_e(\vx) = C_r(\vx) 
+ (\vtau_o - {\vtau_o}_d)^{\tr} \mW_{\vtau_o} (\vtau_o - {\vtau_o}_d)
\end{equation}
and
\begin{equation}
{\vtau_o}_d = \mK_{\vq}( \hat{\vq} - \vq_d ) + \mK_{\dot{\vq}} \hat{\dot{\vq}}
\end{equation}

\section{Operator-Exoskeleton Force Control using position controlled actuation}

At this point our dynamics formulation is no longer appropriate, since we
are assuming the actuators are now position rather than torque sources.

\subsection{Get out of the way control with position sources}

The exoskeleton moves as to create as little force as possible between the
operator and the exoskeleton. The actuators in this case may be position sources or
force sources with high servo gains,
but ultimately operator-exoskeleton forces are mapped
to exoskeleton velocities or angular velocities.

\begin{equation}
\dot{\vq}_d = \mJ^{-1} (\mK_1 \vf)
\label{eq:gootwcps}
\end{equation}
where $\dot{\vq}_d$ are the commanded exoskeleton joint velocities,
$\mJ$ is an appropriate Jacobian matrix, $\mK_1$ is a gain matrix,
and $\vf$ is the force vector to be controlled~\cite{IEEE06990981}.

Inverting a Jacobian matrix is problematic when the matrix is
nearly singular.
Using a fixed or gain scheduled gain matrix may make more sense.
\begin{equation}
\dot{\vq}_d = \mK_3 \vf
\end{equation}
Simplifying equation~\ref{eq:gootwcps}
by assuming the forces to be controlled
are directly applied to the exoskeleton joints, and thus the Jacobian matrix
is the identity matrix, gives us a similar equation:
\begin{equation}
\dot{\vq}_d = \mK_1 \vtau_o
\end{equation}
and we avoid inverting a Jacobian and problems with singularities.

\subsection{Operator force feedback with position sources}

We introduce a desired force to allow for more complex force control.
\begin{equation}
\dot{\vq}_d = \mJ^{-1} (\mK_1 (\vf - \vf_d))
\end{equation}

The exoskeleton could move as to provide a scaled version of exoskeleton-world
contact forces, or some other (usually simple) mapping. 
\begin{equation}
\vf_d = \alpha \mJ_{ow} \vf_w
\end{equation}

\section{Specific Proposed and Implmented Exoskeleton Control Schemes}

\subsection{Sensitivity Amplification Control}

Sensitivity Amplification Control (SAC) used to control BLEEX
is a dynamic cancellation technique (a.k.a.
inverse dynamics, computed torque, or feedback linearization) that also tries
to adjust the apparent inertia of the exoskeleton (equation~\ref{eq:change-inertia1})
using acceleration feedback. It is not clearly stated but it appears that the
acceleration is the result of double differentiating position, as there do not
appear to be velocity sensors on BLEEX. Low pass filtering is applied to reduce
the high frequency noise amplified by the double differentiation process.

\subsection{Integral admittance control}

{\it [Exoskeletons] have the implicit property
of causing a virtual modification of the dynamic response of
the human limb. We use this property of the exoskeletons
action to formulate a unified control design framework called
Integral Admittance [torque to angle] Shaping, which designs exoskeleton con-
trollers capable of producing the desired dynamic response
for the assisted limb. In this framework, a virtual increase
in the admittance of the limb is produced by coupling it
to an exoskeleton that exhibits active behavior. Specifically,
our framework shapes the magnitude profile of the integral
admittance (i.e. torque-to-angle relationship) of the coupled
human-exoskeleton system, such that the desired assistance is
achieved. This framework also ensures that the coupled stability
and passivity are guaranteed.}~\cite{Nagarajan_etal_2015}

{\it ... the impedance of the coupled human-exoskeleton
system needs to be reduced below that of the unassisted
human limb. This implies that the exoskeleton needs to
cancel its own impedance first and then compensate for
at least a part of the human limb’s impedance. Therefore,
the desired exoskeleton behavior must be that of a negative
impedance ... Consequently,
the feedback gains will all be
positive. In other words, the exoskeleton controller uses
positive feedback, and hence the exoskeleton exhibits active
behavior, which is capable of performing net positive work
on the limb. ...
However, positive feedback naturally raises the question of
stability, and so we now explain how coupled stability can be
achieved. Although the exoskeleton exhibits active behavior,
which can be potentially destabilizing, the controller can be
designed such that the coupled human-exoskeleton system
is stable and passive. ... 
}~\cite{Nagarajan_etal_2015}

{\it
... Inertia compensation is more complex ...
It can be shown that using only positive acceleration
feedback ..., the gain margin of the coupled system
reduces to the moment of inertia of the exoskeleton, which
implies that the exoskeleton controller ... can at the most
compensate for the exoskeleton’s own moment of inertia
before going unstable. This implies that the moment of
inertia of the coupled human-exoskeleton system cannot be
reduced below that of the unassisted human limb, without
compromising coupled stability. However, using low-pass
filtered acceleration feedback, it can be shown that inertia 
reduction can be achieved. ...
This work uses filtered acceleration
feedback with a second-order low-pass Butterworth filter. ...
}~\cite{Nagarajan_etal_2015}

This is a form of frequency domain virtual model control.

\begin{verbatim}
Kalman filter:
P. Canet, “Kalman filter estimation of angular velocity and accelera-
tion: On-line implementation,” McGill University, Montr ́eal, Canada,
Tech. Rep. TR-CIM-94-15, Nov. 1994.
\end{verbatim}

\subsection{Dual Control Appproach}

This approach implements a passive swing leg in addition to Sensitivity 
Amplification Control for the stance leg.

{\it The robot utilized the dual-mode control scheme, which is
comprised of the active control for the stance phase and the
passive control (using bypass valves) for the swing phase, to achieve high walking
speed in the swing phase while supporting heavy loads in
the stance phase. To reduce the sudden change of the torque
command at the transition from the swing phase to the stance
phase, a smoothing method is adopted. We also implemented
a pre-transition method to take a foot off quickly for fast
walking by predicting the change from the swing to the stance
in advance.}~\cite{IEEE07222598}

Bypass valves implement what Sarcos calls ``dangle''.

Stance: virtual joint torque control method Kazerooni et al.(2005)
When contact location between the wearer and the
exoskeleton is not fixed and difficult to estimate, this method
has been shown to be an effective method to generate the
locomotion for an exoskeleton robot.[3],[4],[21]
\begin{verbatim}
H. Kazerooni , Z. Racine , L. Huang and R. Steger ”On the control of
the Berkeley lower extremity exoskeleton (BLEEX),” Proc. IEEE Int.
Conf. Robot. Autom., pp. 4364-4371, 2005.
Racine Jean-Louis Charles ”Control of a Lower Extrmity Exoskeleton
for Human Performance Amplification,” University of California,2003.
582
Xiuxia Yang, Hongchao Zhao, Yi Zhang, Xiaowei Liu ”Carrying
Lower Extreme Exoskeleton Rapid Terminal Sliding-Mode Robust
Control,” Journal of computers, vol.9, No.1,202-208. 2014.
\end{verbatim}

To reduce
sudden changes(command jump) at the phase transitions, a
smoothing method [5],[19] is introduced and a pre-transition
method is used to solve the swing delay due to the internal
pressure(approx 5 bar).
\begin{verbatim}
H. Kazerooni , Ryan Steger, Lihua Huang ”Hybrid Control of the
Berkeley Lower Extremity Exoskeleton (BLEEX),” Int. Journal of
Robotics Research, pp. 561-573, 2006.
oonbum Bae, Kyoungchul Kong, Masayoshi Tomizuka ”Gait Phase-
Based Control for a Rotary Series Elastic Actuator Assisting the knee
\end{verbatim}

Transition Control

1) Smoothing Method: During transitions of the gait
phase, discontinuity of the control command torque are
occurred by the different condition for the fixed coordinate
which is on the backpack at the swing phase or the foot at
the stance phase. In this paper, to reduce this sudden change
due to gait phase changes, a smoothing method is proposed
as shown in Fig. 7. An exponential function is considered
as the weighting function for smoothing as shown in (6).
In this case, the weighting is small at the initial stage but
it exponentially converges to one for supporting the load
quickly.

2) Pre-transition Method: The pre-transition method is
that the passive mode is executed in the pre-swing phase
prior to toe off. This dramatically reduces the moving

\subsection{Ground Reaction Force Control}

{\it
The ground reaction
forces (GRF) magnitude and direction is used to command the
actuators. In some researches the GRF sensors are used
together with other sensors in the control architectures [17],
while in other exoskeleton the control system is based on the
GRF merely [RoboKnee, Honda].
}
\begin{verbatim}
17: K. Suzuki, G. Mito, H. Kawamoto, Y. Hasegawa, and Y. Sankai,
"Intention-based walking support for paraplegia patients with Robot
Suit HAL," Advanced Robotics, vol. 21, pp. 1441-1469, 2007
\end{verbatim}

\subsection{Virtual model control}

The goal here is to make the exoskeleton imitate a desired (usually nonlinear)
dynamic system,
For running it could be a pogo stick or trampoline, for example.

There are several possible versions of this type of control:
\begin{enumerate}
\item
Map from operator-exoskeleton contact forces and system state
to exoskeleton actuator or internal forces.
\item
Map from operator-exoskeleton contact forces and system state
to exoskeleton acceleration.
\item
Map from operator-exoskeleton contact forces and system state
to exoskeleton-world contact forces.
\item
Map from exoskeleton-world contact forces and system state to exoskeleton actuator or internal forces.
\item
Map from exoskeleton-world contact forces and system state to exoskeleton acceleration.
\item
Some combination of the above.
\end{enumerate}

\begin{verbatim}
virtual model approach: hopper, compass gait (J. Pratt)
estimate trajectory (jumping) vs. program behavior (trampoline, hopper,
   compass gait)
- J' control + behavior
- Inverse dynamics + behavior
\end{verbatim}

\subsection{Task Specific Control}

Task specific control can involve switching low level control modes, or abstracting
the behavior of the exoskeleton with a hierarchy:
\begin{enumerate}
\item
Task specific control: Generate desired motions and contact forces between
the operator and the exoskeleton, and the exoskeleton and the world.
\item
Sometimes there are intermediate levels of control.
\item
Exoskeleton control: Control the exoskeleton to generate the desired motions and
desired contact forces, with the desired impedance or admittance.
\end{enumerate}

\section{State Estimation}

We are surprised that acceleration feedback based on double differentiation
works well on real systems (even low pass filtered acceleration
estimated by a Kalman filter).
Impacts, shock waves, and the fact that neither
the operator's body parts or the exoskeletons parts are rigid bodies and the
joints are not well defined suggest this approach would be fragile to modeling
error and unmodeled dynamics.
It is not possible to distribute MEMs IMUs throughout the exoskeleton to
improve both velocity and 
acceleration estimation~\cite{Xinjilefu-thesis}, which should improve
Sensitivity Amplification Control.

Role of phase estimation, behavioral clocks, and phase reset events
such as impacts.

\section{Discussion}

Discussion to be written.

\section{Conclusions and Recommendations}

Conclusions and Recommendations to be written.

\bibliographystyle{plain}
\bibliography{exo}

\end{document}


