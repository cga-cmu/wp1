\documentclass[letterpaper,12pt,fullpage]{article}

\usepackage[left=1in,right=1in,top=1in,bottom=1in]{geometry}
\usepackage{cite}
\usepackage{graphicx}
% \usepackage[dvips]{graphicx}
% \usepackage{epsfig} % for postscript graphics files
  % \graphicspath{{../eps/}}
% \DeclareGraphicsExtensions{.eps}
\usepackage{amsmath}
\usepackage{amssymb}
%\usepackage[cmex10]{amsmath}
%\usepackage{array}
%\usepackage{mdwmath}
%\usepackage{mdwtab}
%\usepackage{eqparbox}
\usepackage[tight,footnotesize]{subfigure}
%\usepackage[caption=false]{caption}
%\usepackage[font=footnotesize]{subfig}
%\usepackage{fixltx2e}
%\usepackage{stfloats}
\usepackage{hyperref}

% correct bad hyphenation here
%\hyphenation{op-tical net-works semi-conduc-tor}

\input latex-commands

\begin{document}

\title{A Survey of Near-Term Exoskeleton Technology\\
(Draft 3.0)}

\author{Alex Ansari, Christopher G. Atkeson, Howie Choset, and Matthew Travers\\
Carnegie Mellon University}

\maketitle

\begin{abstract}
Abstract to be written.
\end{abstract}

\section{Executive Summary}

Executive Summary to be written.

\section{Scope: What is this paper about?}

This paper surveys near-term exoskeleton technology with an
emphasis on control.
A companion paper surveys implemented exoskeleton technology [survey.pdf].

The focus is on near-term exoskeleton technologies that allow a
highly trained and top percentile athletic 
operator to carry a payload that weighs approximately the same amount
as the operator. We envisage these types of exoskeletons to be useful
in carrying protective and safety equipment for SWAT teams, police,
firefighters, and soldiers. 

We do not survey exoskeleton technology for human arms. 
We treat the torso and helmet of the exoskeleton as the payload,
and focus on a lower body exoskeleton to support any payloads that
are on the torso or head.

We do not cover global energy storage such as batteries or liquid fuel
or other global power system issues.

A later white paper could discuss the use of propulsive forces other
than foot contacts, as well as additional actuation at the contacts (air
bags, springs, other ways to rapidly apply stored energy) for
operator safety (dodging threats, ...) and high performance (jumping, ...).

A future white paper could discuss near term soft robotics technology
for exoskeletons.

A future white paper could discuss the addition of extra limbs and
other effectors to an exoskeleton~\cite{IEEE07139896}.

%%%%%%%%%%%%%%%%%%%%%%%%%%%%

\section{What are some design philosophy alternatives?}

\subsection{``Invisible'' exoskeleton}

Can we build an exoskeleton allows the operator to behave naturally
and exerts little or no force on the operator?

This design goal
could be achieved through active control using feedback of the
operator-exoskeleton forces and/or high quality prediction of what
the operator will do next. Additional gravity, friction, actuator, and
inertial compensation (inverse dynamics) can improve performance.

\subsection{``Natural'' exoskeleton}

Can we build an exoskeleton allows the operator to behave naturally?
The operator can feel the exoskeleton, but can overpower it when necessary.

One way to achieve this design goal is to have extremely light attachments
to the limbs, and put all heavy exoskeleton components on the
torso and head. The limbs can be physically driven by the operators muscles
with little ``drag'' on the operator.
Bypass valves and clutches could allow the operator to physically ``take over''
and push the relatively lightweight limb mounted parts of the exoskeleton around.

A complementary way to achieve this goal is through active control. Low impedance
(torque source) actuation makes combining active control with operator backdriving
easier.

A ``natural'' exoskeleton is easier to build than 
an ``invisible'' exoskeleton.

\subsection{``Symbiotic'' exoskeleton}

Can we build an exoskeleton that allows the operator/exoskeleton team
to be effective, but not necessarily behave naturally? 
The operator needs to learn to ``fly'' the suit,
just as operators learn to operate parachutes,
wing suits, diving equipment, high and low temperature protective
suits, firefighting equipment, and high altitude flight
and parachute suits.

An advantage of this approach is that 
the exoskeleton can adapt its behavior to the task.
For example, for jumping, the exoskeleton can operate like a pogo 
stick:\\
\url{https://www.youtube.com/watch?v=lbp41vWP4o4},\\ 
and for running it could act like jumping stilts:\\
\url{https://www.youtube.com/watch?v=9ZOd7yEyhwI},

A ``symbiotic'' exoskeleton is easier to build than an ``invisible'' or ``natural''
exoskeleton.


%%%%%%%%%%%%%%%%%%%%%%%%%%%%

\section{What performance is currently available?}

\subsection{Reference Design: linear hydraulic with bypass valves}

\subsubsection{What limits hydraulic performance?}

The differential pressure of the hydraulic fluid supply vs. the fluid
return limits the maximum force. Pressure times piston cross-sectional area
determines actuator force.
Many components are rated for 3000psi = 207bar (usually approximated to
210bar).
(Units: 1 Bar = $10^5$ newtons per square meter (pascals) = 14.5 psi).

The flow capacity of the hydraulic fluid supply and return limits
the maximum speed. Linear velocity*piston cross-sectional area determines 
the fluid flow required.
(Units: 1 lpm = 0.264 gpm)

The power capacity of the hydraulic pump determines the maximum force
at a given flow rate.

To power our SARCOS Primus Humanoid Robot we use an
MTS SilentFlo Hydraulic Power Unit (HPU) 515.11
which operates at 3000psi $\approx$ 210bar,
has a maximum flow of 41.6lpm = 11gpm, and is rated at
18.5 kW = 25hp.

There is also a thermal limit. Hydraulic systems heat up during operation,
and there must be cooling (air or water cooling is often used)
applied somewhere in the system. Heat is transferred by the flowing oil,
so both the HPU, the piping and hoses, and the valves and actuators get hot.

The flow capacity of a hydraulic valve also limits the maximum flow.
The MOOG Type 30 Series 30 Nozzle-Flapper Flow Control Servovalves
are popular on hydraulic humanoids. Their maximum flow is
12lpm = 3.1gpm at 3000psi $\approx$ 210bar.
You might want to get a higher capacity valve. To do this, you
would sacrifice valve bandwidth and flow leakage.
The bandwidth of the MOOG Type 30
Series 30 is about 200Hz, and the leakage ranges from
0.3 to 3gpm. Leakage in this case is internal to the hydraulic
system, from the valve fluid supply port to the valve fluid return port.
Valve leakage is flow and thus power and force
not available to the actuation.
A humanoid robot might have 30 of these valves,
so it would not take a lot of leakage per valve to completely
use up the flow capacity of the HPU.

A linear hydraulic actuator (a piston, for example) has a cross sectional
area ($A$) and a moment arm ($M$). Maximum torque of a single actuator
is $P*A*M$ where $P$ is the operating pressure.
Maximum velocity is $F/(A*M)$ where $F$ is the maximum flow at the
actuator.
Flow, unfortunately, is divided across multiple actuators,
so more moving actuators would reduce this limit.

Here is a simple actuator design:
\begin{equation}
A = 10cm^2 = 10^{-3}m^2, \; \; \; M = 0.01m
\end{equation}
Maximum torque is
\begin{equation}
P*A*M = 210*10^5N*m^{-2}*10^{-3}m^2*0.01m = 210Nm 
\end{equation}
Maximum angular velocity is
\begin{equation}
F/(A*M) = 0.2lps*(1000cm^3/l)/(10cm^2*1cm) = 20r/s
\end{equation}

The Sarcos Primus Humanoid (which uses the MOOG valve described
above) and the Atlas humanoid (which uses similar custom MOOG valves)
are both designed
for maximum joint velocities of about 2 revolutions/second (12.5radians/s).
and generate joint torques of up to around 400Nm.
Doubling the maximum speed would half the maximum torque, which matches
our simple actuator design.

Can we expect a major improvement in hydraulics?
There has been incremental improvement on HPUs and valves since
the 1950s. Rapid prototyping and 3D printing are leading to
easier integration of valves, actuators, and sensors (see following
paragraph). Current high performance are flow control valves. It
is possible that lower leakage valves could be developed. It is possible
that pressure control valves would reduce the need for high bandwidth
force control, and thus high bandwidth valves. Pressure control valves
could also greatly improve exoskeleton safety if pressure control was
implemented physically rather than computationally, since physical
mechanisms are less likely to ``go unstable''.
Bypass mechnisms (a form of clutch) would allow hydraulic systems
to be backdriven without oil flow at high force, which would reduce
the power load on the HPU and the need for high performance force control.
Any valves or additional hydraulic circuitry
would have to have low internal leakage, and previous attempts at
pressure control valves have had leakage problems.
Gear shifts and variable transmissions would allow the maximum torque
and maximum no-load speed of the joint to be increased.

{\bf Integrated Moog hydraulic actuator:}
Moog is developing an integrated hydraulic actuator with built-in
sensors and valves. Our colleagues report:
{\it The integrated Moog system is still very experimental,� there are
only two working prototypes right now.� Our lab is working closely
with Moog to develop the firmware for the builtin controllers. I
think it is a little early to say how well they work, but it looks
promising.}
\url{http://www.designnews.com/document.asp?doc_id=277754&dfpPParams=ht_13,kw_robotics,kw_41,aid_277754&dfpLayout=article}

\subsection{Reference Design: linear electric}

Survey state of the art using linear electric actuators (typically ballscrews):
RoboKnee, Roboray (Samsung)

\subsection{Reference Design: rotary electric}

Survey state of the art using rotary electric actuators:
Schaft~\cite{shaft_foot_placement,shaft_push_recov}, Cheetah(MIT)

Maxon dc flat brushless motor EC45 and servo controller
Elmo. The servomotor emulates the kinematic profile of every
joint and sends a digital signal to the Elmo to form feedbacks.
In order to enhance the impedance capability, harmonic trans-
mission drivers are implemented (model SHD-17-100-2SH
for joints 1 and 2, model SHD-14-100-2SH for joints 3 and 4,
and CSF-32-50-2A-GR for joint 5).~\cite{IEEE07128705}

\subsection{Reference Design: linear pneumatic}

Survey state of the art using linear pneumatic actuators:

\begin{verbatim}
07139975, 07131568, 07251430.pdf
2015 IEEE International Conference on Robotics and Automation (ICRA)
Towards Balance Recovery Control for Lower Body Exoskeleton Robots
with Variable Stiffness Actuators: Spring-Loaded Flywheel Model
Corinne Doppmann, Barkan Ugurlu, Masashi Hamaya, Tatsuya Teramae,
Tomoyuki Noda, and Jun Morimoto
\end{verbatim}

\section{Sensing}

\begin{verbatim}
sensing
 actuator length/angle
 actuator velocity
 actuator force/torque 
n-line loadcells(31E-05KNO,
Sensotec) are equipped on each cylinder tube end for the
actuator force.~\cite{IEEE07222598}
 actuator sensing is pre-transmission
  joint backlash
  joint play
  structural flexibility
  actuator and transmission friction (10 Nm in Atlas)
 joint angle
three high precision(17 bit) encoders(DS-25-16, Net-
zer) are used to measure the robots joint angles(hip, knee,
ankle pitch angle).~\cite{IEEE07222598}
 joint velocity
 joint torque
 IMUs all over
 cameras all over
\end{verbatim}

{\it foot sensors, which are comprised
of four one-axis(normal force) loadcells(CBFSB-200, Bong-
shin), are adopted to detect the gait phase. The placement
of each loadcell is chosen from the analysis of GRF for
level walking : Hallux(toe), first Metatarsal(inside), fourth
Metatarsal(outside), and Heel similar to [20], as shown in
Fig. 2(a). The Gait phase detection is simply classified
into two phases by the threshold method using the sum
of the load cells: Single Stance (Left Stance(LS) or Right
Stance(RS)), Double Stance(DS). The result of gait detection
while walking is shown in Fig. 2(b).}~\cite{IEEE07222598}
20: Kong K, Tomizuka M. ”A gait monitoring system based on air pressure
sensors embedded in a shoe,” IEEE/ASME Trans Mechatron, 358 - 370.
2009.

\section{Energy Storage}

We will only address energy storage
in terms of control issues.

\subsection{Global energy storage}

\begin{verbatim}
 battery
 gasoline/jet fuel/diesel
 chemical (explosive actuation: solid fuel rocket, liquid fuel rocket)
 pressurized gas
 other?
 energy boost systems (jump, escape response in biology)
  chemical (explosive actuation: solid fuel rocket, liquid fuel rocket)
  pressurized gas
  spring
  other?
\end{verbatim}

\subsection{Joint-level/local energy storage (on joint or synergy)}

Can we use local energy storage to assist control: charging batteries,
pressurizing accumulators, using springs, ... ?

{\it incorporation of physical
compliance is of importance when it comes to dependability,
low mechanical impedance, locomotion efficiency, inherent
safety and enhanced environmental interaction capabilities}~\cite{IEEE07139975}

\begin{verbatim}
 accumulator (really an air spring)
 physical springs/damping
 supercapacitor/battery/electrical for local electric braking
\end{verbatim}

\section{Actuation}

Survey possible actuation systems.

What role do physical properties of actuation play in control? Highly
geared electric systems and hydraulic systems are inherently very
stiff and go where you want, but must be made compliant/backdrivable
using high performance (expensive and fragile) force control.
Pneumatics and Series Elastic Actuation (SEA) are inherently
compliant/backdrivable, which may reduce the performance
needed from the sensors and control system, but may not be able to
produce needed forces and speeds.

\subsection{Global actuation}

\begin{verbatim}
 hydraulic pump 
 gas (air) pump
\end{verbatim}

\subsection{Joint-level/synergy/local actuation}

\begin{verbatim}
 rotary electric
  motor
 linear electric
  ball screw (both motor and transmission)
 linear hydraulic
  moog thing
 rotary hydraulic
 linear pneumatic
 rotary pneumatic
 linear or rotary impulsive actuation (IC engine style actuation)
 energy boost systems (jump, escape response in biology)
  chemical (air bag style: explosive actuation: solid/liquid fuel rocket)
  pressurized gas
  spring
  other?
\end{verbatim}

\subsection{Flywheels}

Energy storage and actuation for high manueverability.

\subsection{Hybrid actuation}

Different types of actuation can be combined: springs and electric motors,
for example. This combination can also be viewed as combining energy
storage and actuation. Slow pneumatics and fast electric motors~\cite{Morimoto,UCLA}.
Mini/Macro~\cite{Stanford}. In this view, the human operator is just
one more set of actuators to be added to the combined system.

\begin{verbatim}
parallel combinations - weak fast actuator needs to hit joint limit and then be strong.
 (leads to chattering) or just lock in position to exert big force
series combination - strong actuator needs to clutch out of circuit to move fast
\end{verbatim}

\section{Transmissions and Drive Trains}

\subsection{How can clutches and gear shifts help?}

How can we implement a gear shifting system (the crucial mechanical
system missing in current robots)? How do we switch from manipulation
while standing, walking, running, other rapid movement (dodging), and
power movements (jumping, moving through doors and walls)?

Stance leg requires a large torque and
relatively low bandwidth during the support period. However,
the swing leg undergoes a large motion, but it only supports
its own weight during the swing period. Therefore, the swing
leg requires a relatively small torque. Thus, a dual-mode
approach is very useful for an exoskeleton robot control.
In this paper, a dual-mode control method using a hydraulic
system is proposed,~\cite{IEEE07222598}

Rise from squat, chair.
push large objects out of the way (standing full body push
or seated using legs to push)

\subsection{Global Transmission}

\begin{verbatim}
 electric circuit
 hydraulic circuit
 gas (air) circuit
\end{verbatim}

\subsection{Joint-level/synergy/local transmission}

\begin{verbatim}
 rigid
  gears
   planetary
   other
  harmonic drive
  ball screw
  cosine attachment
  Sarcos Primus Humanoid knee - to reduce moment arm variation with angle
 Bowden cable
 tendon drives (can go slack)
 hydraulic/pneumatic circuit
 variable
  belts
  cones
  fluid
  adjustable moment arm
   linear slider
   XY slider for actuation and ``joint'' on foot.
 series elastic actuation
\end{verbatim}

\section{Address what is the complexity penalty, weight penalty, 
and ``drag'' of actuator on individual joint motion.}

\section{System issues}

Things that don't fit well elsewhere.

\begin{verbatim}
 bearing play, structural flexibility: +/- 2cm COM uncertainty
 how attached to human (exoskeleton)
 air bags
\end{verbatim}

The human and exoskeleton limbs are typically connected to each
other with straps or padding. These connections typically have substantial
compliance and play, and may or may not have appropriate damping.
Typically the human and the exoskeleton do not move the same way:
pose, and linear and angular velocities and accelerations are not the same.

Heat exhaust. Heatsink cooling, Force air cooling, Water cooling.
Will housing and structure get too hot and fry operator?

\section{Discussion}

{\bf Weight distribution:}
There is a weight distribution issue. Ideally the exoskeleton should
have a center of mass at the same location as the operator. Otherwise,
operator movements can be degraded and substantial forces occur between
the operator and the exoskeleton.
This constraint can be reduced with the use of propulsive forces other
than foot contacts: rockets (solid or liquid fuel), jets, exploding armor,
ducted fan air flow, ...

{\bf One Time Use:}
What happens if the suit is "one time use"? For example, ablative armor,
armor that reacts to incoming projectiles before they hit,
or explosive shielding could be used. Could the armor weight be decreased?
The expensive parts of the suit could be re-usable, and even autonomously
return to safe areas.

{\bf Do we need physical contact?}
Current exoskeletons use physical contact with the operator to control
the exoskeleton. Can we make an exoskeleton that does not touch or
minimally touches the operator, and in which there is no power
transfer between the operator and the exoskeleton? These types of
systems might allow an operator to behave more normally, have a
greater range of movement, be less tiring, and for non-enclosing
systems, allow the operator to move independently of the system (such
as aim and fire standard weapons or dive for cover) for additional
operator performance and safety.
\begin{enumerate}
\item
A "fat suit" or "sumo suit" could fully enclose the operator, but
use distance sensing to move with the operator.
\item
A "shield" could enclose the front of an operator (or whatever
portion of the operator is threatened) and move with the operator,
\item
A "human shield" (actually a machine shield) could move ahead of
an operator, mimicking the operators movement.
\item
An "angel" could walk, jump, or fly between an operator and potential
hostile sites, and deflect or disable projectiles.
\end{enumerate}

\section{Conclusions and Recommendations}

1) We recommend the exploration of how to implement and utilize practical
gear shifts and clutches (or equivalents). For hydraulics and air, this might mean
bypass or other special valve features.
For electric motors and gears, this might mean special purpose clutch mechanisms
built into gearing or harmonic drives. For linear actuators, antagonistic actuation
combined with tendons that can go slack can be used for clutching.

2) We recommend exploring the combination of passive energy storage (springs,
accumulators, supercapacitors, ...) with active actuation. One form of this
is series elastic actuation. 

3) We also recommend exploring storing energy with
a ``catch'' and ``explosive release'' for overdrive capabilities. 

4) We recommend the exploration of hybrid actuation, where slow and strong actuators
are combined with fast but weak actuators either in parallel or serial.

5) We recommend the exploration of multi-joint or synergy actuation, where an actuator
might cross two or more joints.

6) We recommend the exploration of integrated actuation units using 3D printing
or other rapid prototyping techniques, such as the integrated Moog hydraulic
actuator for HyQ.

7) We recommend applying as many sensors as possible, and asking the performers
to make it easy to add more by making the sensor network available in the design.
This maximizes the probability of success
by enabling multiple control
approaches to be implemented, refined, and support each other.
In particular, we would like to see force sensing between the operator and the
exoskeleton, including at the feet, as much force sensing between the exoskeleton 
and the world as possible, but minimally full six-axis force/torque sensing at
the exoskeleton feet (where they touch the ground). We would like to see multiple MEMs
IMUs (measuring linear acceleration and angular velocity) installed across 
the exoskeleton and at least one high quality fiber optic gyro IMU.
We would like to see high quality direct velocity sensing
such as high count (100,000 counts/revolution) encoders and/or 
analog rotary or linear tachometers on actuators and joints. 
We would like to see actuator force or torque sensors
(load cells or equivalents) with the measurement on the link side (rather than
the actuator side) of any transmission. Electric current in motors and oil
pressure in hydraulic pistons can also be used for actuator force estimation,
but because these measurements are on the actuator side before the transmission
and in the case of hydraulics before the oil seals on the piston, these measurements
are greatly contaminated by friction. On the Atlas humanoid we typically saw 10Nm
joint torque estimation errors for a system that estimated actuator output using
oil pressure on each side of the piston head.
A possible objection to this is additional cost. We feel it is a false
economy to skimp on sensing. Leaving practical sensing out greatly increases
the risk of poor performance.

8) Do we recommend flywheels?

More Conclusions and Recommendations to be written.

\bibliographystyle{plain}
\bibliography{exo}

\end{document}


