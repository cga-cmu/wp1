\documentclass[letterpaper,12pt,fullpage]{article}

\usepackage[left=1in,right=1in,top=1in,bottom=1in]{geometry}
\usepackage{cite}
\usepackage{graphicx}
% \usepackage[dvips]{graphicx}
% \usepackage{epsfig} % for postscript graphics files
  % \graphicspath{{../eps/}}
% \DeclareGraphicsExtensions{.eps}
\usepackage{amsmath}
\usepackage{amssymb}
%\usepackage[cmex10]{amsmath}
%\usepackage{array}
%\usepackage{mdwmath}
%\usepackage{mdwtab}
%\usepackage{eqparbox}
\usepackage[tight,footnotesize]{subfigure}
%\usepackage[caption=false]{caption}
%\usepackage[font=footnotesize]{subfig}
%\usepackage{fixltx2e}
%\usepackage{stfloats}
\usepackage{hyperref}

% correct bad hyphenation here
%\hyphenation{op-tical net-works semi-conduc-tor}

\input latex-commands

\newcommand{\T}{{\mbox {T}}}

\newcommand{\gc}{{\mbox {GravityCompensation}}}

\newcommand{\shrinkfig}{\def\baselinestretch{1.0}\small} % 0.9 okay
\newcommand{\shrink}{\def\baselinestretch{1.1}\small} % 0.97 % 0.95 okay

\begin{document}

\title{A Survey of Exoskeleton Control (Draft 1.1)}

\author{Alex Ansari, Christopher G. Atkeson, Howie Choset, and Matthew Travers\\
Carnegie Mellon University}

\maketitle

\begin{abstract}
Abstract to be written.
\end{abstract}

\section{Executive Summary}

Executive Summary to be written.

\section{What is this paper about?}

This paper surveys state of the art exoskeleton technology with an
emphasis on control. The focus is on exoskeleton technologies that allow a
highly trained and top percentile athletic 
operator to carry a payload that weighs approximately the same amount
as the operator. We envisage these types of exoskeletons to be useful
in carrying protective and safety equipment for SWAT teams, police,
firefighters, and soldiers. 

We expect each exoskeleton controller
to be used by and optimized for a single operator.
A substantial investment in capturing the operators normal behavior,
operator training and learning, and controller customization can be made.

We do not survey exoskeleton technology for human arms. 
We treat a torso and helmet of the exoskeleton as the payload,
and focus on a lower body exoskeleton to support any payloads that
are on the torso or head.
We do not cover global energy storage such as batteries or liquid fuel.

\section{What would it be like for the operator?}

\subsection{``Invisible'' exoskeleton}

Can we build an exoskeleton allows the operator to behave naturally
and exerts little or no force on the operator?

This could be achieved through active control using feedback of the
operator-exoskeleton forces. Additional gravity, friction, actuator, and
inertial compensation (inverse dynamics) can improve performance.

\subsection{``Natural'' exoskeleton}

Can we build an exoskeleton allows the operator to behave naturally?
The operator can feel the exoskeleton, but can overpower it when necessary.

Bypass valves and clutches allow the operator to physically ``take over''
and push the relatively lightweight leg mounted exoskeleton around.

``Passive'' degrees of freedom can have physical or ``virtual'' springs
applied to assist in support (all abduct/adduct DOF, rotational DOF).
These aids can be clutched out or turned off when necessary.
``Virtual'' sprintgs can be modulated.

\subsection{``Symbiotic'' exoskeleton}

Can we build an exoskeleton that allows the operator/exoskeleton team
to be effective? The operator needs to learn to ``fly'' the suit,
just as operators learn to operate parachutes,
wing suits, diving equipment, high and low temperature protective
suits, firefighting equipment, and high altitude flight suits.

The exoskeleton can adapt its behavior to the task.
For example, for jumping, the exoskeleton can operate like a pogo 
stick \url{https://www.youtube.com/watch?v=lbp41vWP4o4}, 
and for running it could act like jumping 
stilts \url{https://www.youtube.com/watch?v=9ZOd7yEyhwI},

\subsection{Physical Constraints}

1) There is a weight distribution issue. Ideally the exoskeleton should
have a center of mass at the same location as the operator. Otherwise,
operator movements can be degraded and substantial forces occur between
the operator and the exoskeleton.

This constraint can be reduced with the use of propulsive forces other
than foot contacts: rockets (solid or liquid fuel), jets, exploding armor,
ducted fan air flow, ...

A later white paper will discuss the use of propulsive forces other
than foot contacts, as well as additional actuation at the contacts (air
bags, springs, other ways to rapidly apply stored energy) for
operator safety (dodging threats, ...) and high performance (jumping, ...).

\section{Recognizing Operator Intent}

Survey current operator/exoskeleton interfaces.

The further in the future the control system can predict what the operator
is going to do, the easier it is to control the exoskeleton. One way
to think about predicting operator intent is to imagine you are in
a hand-to-hand fight and you want to predict your opponent's next move.
You can also think about playing poker. What ``tells'' does the opponent
have?
You are allowed to instrument your opponent. What sensors would you use?
How would you interpret the sensors?

Here is an example of a robot seeing and moving faster than a human, and
consistantly beating a human opponent in rock-paper-scissors.
\url{https://www.youtube.com/watch?v=Qb5UIPeFClM&feature=player_embedded}

\subsection{Intent time scales}

What time scales does intent operate on, and what might we use
to estimate intent on different time scales?

\begin{verbatim}
10 sec - rely on perceiving situation and thinking like your opponent.
         What are the probabilities of various attacks? Responses?
1 sec - measure set/tells to estimate probability of particular behaviors.
100 msec - EMG, muscle state, force helpful here
10 msec - EMG, muscle state, force helpful here and
         immediate sensors: joint pos,vel, IMU, contact forces
1 msec - immediate sensors: joint pos,vel, IMU, contact forces
\end{verbatim}

\subsection{Signal flow analysis}

To what extent can we use brain, neural, and muscle electrical signals
(EMG) to anticipate what the operator will do and increase performance?

\subsubsection{What signals are there?}

\begin{verbatim}
- perceive what user sees, hears, feels, smells, tastes.
- superhuman perception (RF signals, UV, IR, ultrasound, ...)
- brain signals
- spinal signals (motorneuron pools)
- motor nerve signals
- muscle electrical signals (EMG)
- muscle force signals (FMG), muscle force at tendons, muscle internal pressure
- sensory nerve signals
- implanted tissue markers, other implanted sensors
- tissue imaging (ultrasound, optical)
- human/exo contact force
- contact strain, deformation rate
- exo joint/IMU sensing
- train user to emit special signals (play a videogame with their body,
  hands, eyes, neural signals, muscle signals, ...)
\end{verbatim}

A future white paper will discuss how intent prediction could be 
made easier with superhuman perception.

A future white paper will discuss how intent prediction could be 
made easier with implanted markers and devices, as well as continuous
operator imaging such as ultrasound.

\subsubsection{Signal timing?}

So for inferring the ``intent'' of a human operator of an
exoskeleton,
I would like to put a timeline together of events.
I guessed some timing, which I hope you will correct.

When I say leg EMG or leg muscle I mean
gastrocneumius or vastus lateralis (``fast'' muscles).

What are delays between:
\begin{verbatim}
visual event to leg EMG start
auditory event to leg EMG start
vestibular event to leg EMG start
foot tactile event to leg EMG start
EMG start to 10% leg muscle
EMG start to peak leg muscle
\end{verbatim}

What are your guesses for motorneuron activation to leg EMG activation?

{\b From Hartmut Geyer:}

The traveling time for neural signals is roughly 100m/s in humans.
Thus, to go from any sensory system in the head (vestibular, ocular,
auditory) to the leg muscles will take between 50ms (hip muscles) and
100ms (foot muscles). Once the signal reaches the muscle, it takes
another 5-10ms to be recognized by the surface EMG sensor (electrical
field propagation in muscle tissue is about 0.4m/s).

With that calculation:\\
vis/aud/vestib to EMG: 50-70ms for vastus and about 80-100ms for
gastroc

example: paper on vestibular reflexes reporting 60ms to
90ms:�\url{http://www.ncbi.nlm.nih.gov/pmc/articles/PMC4288134/}
��
If the motor cortex is involved with active calculations, times can
take longer (I don't know numbers for this, but would guess adding
50-100ms). I've heard that in combat soldiers, the motor cortex is
getting by-passed, and they act far more reflexively from sensory
systems directly through the lower brain. So if this is for military
applications, I'd expect to see responses mainly determined by
reflex travel times.

Foot tactile event to gastrocnemius EMG:
\begin{itemize}
\item
traveling time in neural system: 10ms foot sensors->spinal cord +
10ms spinal cord->muscle�
\item
mechanical delay foot to foot sensory cells: 5-10ms
\item
electrical delay muscle activation to EMG: 5-10ms
\item
total: 30-40ms (which is consistent with reported EMG activities
after mechanical disturbance at ankle joint with bowden cable jerk:
\url{http://jn.physiology.org/content/76/2/1112.short�reports 42ms)}
\end{itemize}

Foot tactile event to vastus EMG:\\�
- same as gastrocnemius less about 5ms for shorter travel time down
the spinal cord to the muscle.

EMG start to 10\% leg muscle force:\\
- this gets tricky: EMG has fixed delay from muscle activation due to
travel of electrical field to muscle surface (5-10ms, see top). Delay
of muscle force production follows low pass filter function (calcium
ion dynamics), often modeled as single pole filter with 10ms
characteristic time. At 10\% force, both EMG and leg force could
actually occur at the same time.�

EMG start to peak muscle force:\\
typically, (full) force production trails EMG by about 30ms in fast
muscles (electromechanical delay, due to calcium ion dynamics
traveling from sarkoplasmatic reticulum to cross bridge binding
sites, \url{http://www.ncbi.nlm.nih.gov/pubmed/527577})

alpha motoneuron activation to leg EMG:\\
- travel time spinal cord to muscle (5ms for vastus, 10ms for
gastocnemius) plus 5-10ms before muscle activation gets recognized in
surface EMG:\\
net vastus: 10-15ms\\
net gastrocnemius: 15-20ms

\subsubsection{What can be measured or estimated?}

Predictive models of human motion, a major topic of robot learning
from observation/demonstration/imitiation and computer animation,
will be the subject of a future white
paper~\cite{IEEE06913830,Bagnell}

FMG-[06913842]

{\bf EMG}
Electrical activity was measured using surface electrodes,
high-pass filtered at 20 Hz, rectified, low-pass filtered at 6 Hz,
offset by a small value of -0.008, and amplified by a gain of 283
to obtain the desired torque.~\cite{IEEE07139980}

Electromyography (EMG)
signals are extremely noisy and it is not easy to extract a
meaningful data form the raw signal. In addition, it takes a long
time to attach the EMG electrodes to the body and calibrate the
sensors.
\begin{verbatim}
T. Hayashi, H. Kawamoto, and Y. Sankai, "Control method of robot
suit HAL working as operator's muscle using biological and dynamical
information," in Intelligent Robots and Systems, 2005.(IROS 2005).
2005 IEEE/RSJ International Conference on, 2005, pp. 3063-3068.
H. Kawamoto and Y. Sankai, "Power assist system HAL-3 for gait
disorder person," in Computers helping people with special needs, ed:
Springer, 2002, pp. 196-203.
\end{verbatim}

muscle moves -> surface electrodes come off (especially for large fast
movements, the worst time).
perspration changes measurement. needs to be calibrated for each application.

{\bf Foot contact and force/torque sensing:}
\begin{verbatim}
[8] K. Kong and M. Tomizuka, “A Gait Monitoring System Based on Air
Pressure Sensors Embedded in a Shoe,” IEEE/ASME Transactions on
Mechatronics, vol. 14, pp. 358–370, 2009.
\end{verbatim}

\subsubsection{How can the operator communicate with the exoskeleton}

\begin{verbatim}
Operator uses high frequency range or patterns of output to
signal what to do
 - selection
 - parameterization (how far, how high, how fast)
\end{verbatim}

The operator controls more than the pose and motion of the
exoskeleton. What if there are "pointing" sensors or communication
devices that need to be aimed at or track an area of interest? What if
there are additional (physical or virtual) pan/tilt/zoom cameras
pointing to the side and rear or line of sight secure communications,
for example. What about other controls, such as power assist level or
thermal control? How can an operator naturally express intent and
control these additional degress of freedom?

\subsection{Operator load and fatigue}

To what extent can task recognition and "autonomy" reduce operator
load and fatigue?

\subsection{References}

\begin{verbatim}
H.Kawamoto and Y.Sankai, “Power Assist Method Based on Phase
Sequence Driven by Interaction between Human and Robot Suit,”
Proceedings of the IEEE International Workshop on Robot and Human
Interactive Communication, pp. 491–496, 2004.
\end{verbatim}

\section{What are the specs?}

Hip, knee, ankle: At least 200Nm, ideally >400Nm.

\subsection{What speeds and forces are needed to walk?}

\subsection{What speeds and forces are needed to run?}

\subsection{What operator-exoskeleton contact forces are acceptable?}

\section{What performance is currently available?}

\subsection{Reference Design: linear hydraulic with bypass valves}

Survey state of the art using linear hydraulic actuators:
Sarcos Primus Humanoid, Atlas(BDI), Cheetah(BDI)

{\it 140 bar(2,050 psi) and 8 l pm in total. Hydraulic cylinders
are designed to be double-acting at the hip pitch joint and
the knee pitch joint. The hip and knee cylinders are designed
for a maximum thrust of 4 kN and 2 kN, respectively. The
four three-way servo valves(M 200, Star-Hydraulic)}~\cite{IEEE07222598}


\subsection{Reference Design: linear electric}

Survey state of the art using linear electric actuators (typically ballscrews):
Roboray (Samsung)

\subsection{Reference Design: rotary electric}

Survey state of the art using rotary electric actuators:
Schaft~\cite{shaft_foot_placement,shaft_push_recov}, Cheetah(MIT)

Maxon dc flat brushless motor EC45 and servo controller
Elmo. The servomotor emulates the kinematic profile of every
joint and sends a digital signal to the Elmo to form feedbacks.
In order to enhance the impedance capability, harmonic trans-
mission drivers are implemented (model SHD-17-100-2SH
for joints 1 and 2, model SHD-14-100-2SH for joints 3 and 4,
and CSF-32-50-2A-GR for joint 5).~\cite{IEEE07128705}

\subsection{Reference Design: linear pneumatic}

Survey state of the art using linear pneumatic actuators:

\begin{verbatim}
07139975
2015 IEEE International Conference on Robotics and Automation (ICRA)
Towards Balance Recovery Control for Lower Body Exoskeleton Robots
with Variable Stiffness Actuators: Spring-Loaded Flywheel Model
Corinne Doppmann, Barkan Ugurlu, Masashi Hamaya, Tatsuya Teramae,
Tomoyuki Noda, and Jun Morimoto
\end{verbatim}

\section{Sensing}

\begin{verbatim}
sensing
 actuator length/angle
 actuator velocity
 actuator force/torque 
n-line loadcells(31E-05KNO,
Sensotec) are equipped on each cylinder tube end for the
actuator force.~\cite{IEEE07222598}
 actuator sensing is pre-transmission
  joint backlash
  joint play
  structural flexibility
  actuator and transmission friction (10 Nm in Atlas)
 joint angle
three high precision(17 bit) encoders(DS-25-16, Net-
zer) are used to measure the robots joint angles(hip, knee,
ankle pitch angle).~\cite{IEEE07222598}
 joint velocity
 joint torque
 IMUs all over
 cameras all over
\end{verbatim}

{\it foot sensors, which are comprised
of four one-axis(normal force) loadcells(CBFSB-200, Bong-
shin), are adopted to detect the gait phase. The placement
of each loadcell is chosen from the analysis of GRF for
level walking : Hallux(toe), first Metatarsal(inside), fourth
Metatarsal(outside), and Heel similar to [20], as shown in
Fig. 2(a). The Gait phase detection is simply classified
into two phases by the threshold method using the sum
of the load cells: Single Stance (Left Stance(LS) or Right
Stance(RS)), Double Stance(DS). The result of gait detection
while walking is shown in Fig. 2(b).}~\cite{IEEE07222598}
20: Kong K, Tomizuka M. ”A gait monitoring system based on air pressure
sensors embedded in a shoe,” IEEE/ASME Trans Mechatron, 358 - 370.
2009.

\section{Energy Storage}

We will only address energy storage
in terms of control issues.

\subsection{Global energy storage}

\begin{verbatim}
 battery
 gasoline/jet fuel/diesel
 chemical (explosive actuation: solid fuel rocket, liquid fuel rocket)
 pressurized gas
 other?
 energy boost systems (jump, escape response in biology)
  chemical (explosive actuation: solid fuel rocket, liquid fuel rocket)
  pressurized gas
  spring
  other?
\end{verbatim}

\subsection{Joint-level/local energy storage (on joint or synergy)}

Can we use local energy storage to assist control: charging batteries,
pressurizing accumulators, using springs, ... ?

{\it incorporation of physical
compliance is of importance when it comes to dependability,
low mechanical impedance, locomotion efficiency, inherent
safety and enhanced environmental interaction capabilities}~\cite{IEEE07139975}

\begin{verbatim}
 accumulator (really an air spring)
 physical springs/damping
 supercapacitor/battery/electrical for local electric braking
\end{verbatim}

\section{Actuation}

Survey possible actuation systems.

What role do physical properties of actuation play in control? Highly
geared electric systems and hydraulic systems are inherently very
stiff and go where you want, but must be made compliant/backdrivable
using high performance (expensive and fragile) force control.
Pneumatics and Series Elastic Actuation (SEA) are inherently
compliant/backdrivable, which may reduce the performance
needed from the sensors and control system, but may not be able to
produce needed forces and speeds.

\subsection{Global actuation}

\begin{verbatim}
 hydraulic pump 
 gas (air) pump
\end{verbatim}

\subsection{Joint-level/synergy/local actuation}

\begin{verbatim}
 rotary electric
  motor
 linear electric
  ball screw (both motor and transmission)
 linear hydraulic
  moog thing
 rotary hydraulic
 linear pneumatic
 rotary pneumatic
 linear or rotary impulsive actuation (IC engine style actuation)
 energy boost systems (jump, escape response in biology)
  chemical (air bag style: explosive actuation: solid/liquid fuel rocket)
  pressurized gas
  spring
  other?
\end{verbatim}

\subsection{Flywheels}

Energy storage and actuation for high manueverability.

\subsection{Hybrid actuation}

Different types of actuation can be combined: springs and electric motors,
for example. This combination can also be viewed as combining energy
storage and actuation. Slow pneumatics and fast electric motors~\cite{Morimoto,UCLA}.
Mini/Macro~\cite{Stanford}. In this view, the human operator is just
one more set of actuators to be added to the combined system.

\begin{verbatim}
parallel combinations - weak fast actuator needs to hit joint limit and then be strong.
 (leads to chattering) or just lock in position to exert big force
series combination - strong actuator needs to clutch out of circuit to move fast
\end{verbatim}

\section{Transmissions and Drive Trains}

\subsection{How can clutches and gear shifts help?}

How can we implement a gear shifting system (the crucial mechanical
system missing in current robots)? How do we switch from manipulation
while standing, walking, running, other rapid movement (dodging), and
power movements (jumping, moving through doors and walls)?

Stance leg requires a large torque and
relatively low bandwidth during the support period. However,
the swing leg undergoes a large motion, but it only supports
its own weight during the swing period. Therefore, the swing
leg requires a relatively small torque. Thus, a dual-mode
approach is very useful for an exoskeleton robot control.
In this paper, a dual-mode control method using a hydraulic
system is proposed,~\cite{IEEE07222598}

Rise from squat, chair.
push large objects out of the way (standing full body push
or seated using legs to push)

\subsection{Global Transmission}

\begin{verbatim}
 electric circuit
 hydraulic circuit
 gas (air) circuit
\end{verbatim}

\subsection{Joint-level/synergy/local transmission}

\begin{verbatim}
 rigid
  gears
   planetary
   other
  harmonic drive
  ball screw
  cosine attachment
  Sarcos Primus Humanoid knee - to reduce moment arm variation with angle
 Bowden cable
 tendon drives (can go slack)
 hydraulic/pneumatic circuit
 variable
  belts
  cones
  fluid
  adjustable moment arm
   linear slider
   XY slider for actuation and ``joint'' on foot.
 series elastic actuation
\end{verbatim}

\section{System issues}

Things that don't fit well elsewhere.

\begin{verbatim}
 bearing play, structural flexibility: +/- 2cm COM uncertainty
 how attached to human (exoskeleton)
 air bags
\end{verbatim}

The human and exoskeleton limbs are typically connected to each
other with straps or padding. These connections typically have substantial
compliance and play, and may or may not have appropriate damping.
Typically the human and the exoskeleton do not move the same way:
pose, and linear and angular velocities and accelerations are not the same.

Heat exhaust. Heatsink cooling, Force air cooling, Water cooling.
Will housing and structure get too hot and fry operator?

\section{Control approaches}

What are current approaches to exoskeleton control?

How do exoskeleton control approaches compare with current and
possible approaches to mobile manipulation and humanoid robot control?

Role of phase estimation, behavioral clocks, and phase reset events
such as impacts.

One basic distinction is whether the actuators are thought of as position
or force sources. Often this is revealed by the design of the ``low level''
controller. Note that this is often somewhat confusing. Electric motors
are torque sources (low impedance), but when a high gear ratio transmission
is added (such as a harmonic drive) the whole system becomes high impedance
and thus is best thought of as a position (or velocity) source. Therefore,
position control is performed by the low level controller. Hydraulic actuators
are high impedance, but when a force or load sensor is added (or piston differential
oil pressure is used as a force sensor), the whole system becomes low impedance
and is best thought of as a force source. Force (or joint torque) control is
performed by the low level controller. Force or torque can be added to an
electromechanical drive train (electric motor plus gears or motor plus ballscrew)
to make it lower impedance and more like a force source as well.

Terms like impedance and admittance control are used, but are often confusing,
as in controller design one can choose from several possible
inputs into the exoskeleton (exoskeleton positions,
velocities, accelerations, operator-exo contact forces and
exo-world contact forces), and choose from several possible exoskeleton
outputs: exoskeleton actuator forces and torques, exoskeleton
motion (position, velocity, and acceleration), as well as operator-exo
contact forces or exo-world contact forces.

Useful paper: \url{http://summerschool.stiff-project.org/fileadmin/pdf/1804_C19.pdf}

Variants of nonlinear feedback control such as feedback linearization or
sliding mode control are largely ignored. Once the decision to use feedback control
based on a set of observable quantities and with particular outputs is made,
one can try out the various linear and nonlinear feedback control paradigms
to see what works well.
Variants of function approximation methods such as lookup tables, fuzzy logic,
sigmoidal neural networks, radial basis functions, and locally weighted regression
are also largely ignored. Once the decision to use a function approximator
and what the inputs and outputs has been made, one can try out the various
approaches to see what works well.
Same for optimization methods.
Same for constraint enforcement (such as avoiding self-collisions)
in either optimization or feedback control (barrier
Lyapunov functions (BLF)~\cite{IEEE06911561}).

Stability proofs of any of these methods should be viewed extremely skeptically due
to the unmodeled operator-exoskeleton and exoskeleton-world contact dynamics, 
actuator unmodeled dynamics, joint play and structural deformation, controller
time delays which are typically ignored, especially in proofs involving
passivity arguments or Lyapunov functions.

The dynamics of a revolute exoskeleton are:
\begin{equation}
\mM(\vq) \ddot{\vq} + \mC(\vq,\dot{\vq}) + \mG(\vq) + \mF(\vq,\dot{\vq})
+ \mJ_w^{\tr}(\vq) \vf_w = \vtau + \mJ_o^{\tr}(\vq) \vf_o
\end{equation}
where $\vq$ are the joint angles and position and velocity of the ``root'' of
the exoskeleton, $\mM$ is the inertia matrix, $\mC()$ are the Coriolis and
centripetal forces, $\mG()$ are the gravitational forces, $\mF$ are the
friction forces, $\mJ_w^{\tr}(\vq) \vf_w$ (Jacobian matrix multiplied with the
contact forces)
are the exoskeleton-world
contact forces 
expressed as exoskeleton joint torques, 
$\vtau$ are the exoskeleton
joint torques,
and $\mJ_o^{\tr}(\vq) \vf_o$
are the operator-exoskelton forces expressed
as exoskeleton joint torques,

To understand these equations better, it is useful to have all Jacobians equal
to the identity matrix. This means the operator is directly applying torques
at the exoskeleton joints, and so is the world. In this case:
\begin{equation}
\mM(\vq) \ddot{\vq} + \mC(\vq,\dot{\vq}) + \mG(\vq) + \mF(\vq,\dot{\vq})
+ \vtau_w = \vtau + \vtau_o
\end{equation}

Note that we are ignoring the operator dynamics and operator muscle and tissue
forces resulting in operator joint torques. We will add these in later (???).
We are also ignoring actuator dynamics, which we can handle by augmenting 
the state vector.

\subsection{No control}

Rely on counterbalancing, mechanical springs, air springs, etc.
to reduce forces. 
Operator relies on mechanical backdrivability to drive
exokeleton to do task.

$\vtau$ is generated by passive devices such as springs or dampers.

\subsection{Closed loop behavior selection, open loop behavior
execution}

Use joint/Cartesian position and velocity control of exoskeleton 
with pre-generated references or selected targets and automatically generated trajectories to move to targets or perform tasks.
Getting behavior selection is very important, operator may do it with
special interface (keyboard, game controller, ...).
Operator relies on mechanical backdrivability to correct exoskeleton behavior.

Recognizing/estimating phase goes here.

Phase Triggered References could go here.

Motion capture, motion libraries goes here.

Often finite state machines whose states correspond to left stance, double support 1,
right stance, and double support 2 are used to select behaviors. Gait transitions
can be triggered by foot touchdown and liftoff events, or
displacement of the Center of Mass (CoM).

step-width adaptation algorithm

$\vtau = \vtau_{ff}(t)$ where $\vtau_{ff}$ is a time
dependent feedforward torque vector.

\subsection{Active gravity compensation}

Similar to counterbalancing or physical gravity compensation, except done
with exoskeleton actuators to cancel exoskeleton weight (but not inertia).
A simple formulation is:
\begin{equation}
\vtau = \mJ^{\tr} (m\vg)
\end{equation}
where $\vtau$ are the exoskeleton joint torques, 
$\mJ$ is a Jacobian matrix, $m$ is the weight of the exoskeleton,
and $\vg$ is the gravity vector. 
The operator relies on mechanical backdrivability to drive
the exokeleton to do tasks.

Sarcos Primus Humanoid (ATR) video:\\
\url{https://www.youtube.com/watch?v=KNxxLm4sPys}

$\vtau = \hat{\mG}(\hat{\vq})$ 
are the estimated gravitational torques at the estimated exoskeleton configuration.

\subsection{Friction compensation}

$\vtau = \hat{\mF}(\hat{\vq},\hat{\dot{\vq}})$ 
are the estimated friction torques at the estimated exoskeleton configuration
and velocity.
Note that these may be load dependent, and may require additional state to
correctly handle stiction and hysteresis effects.

\subsection{Online inverse dynamics control}

The goal here is to map from desired motion (in this case desired
accelerations) to exoskeleton actuator commands (and desired operator
and world contact forces).
The exoskeleton actuator torque 
attempts to match the torques caused by inertial, Coriolis, centripetal,
gravitational, frictional, and exoskeleton-world contact forces.
It relies on estimated dynamic models and their parameters.
If there are actuator dynamics or other dynamics, we can include those
dynamics as well.
\begin{equation}
\vtau = \vtau_{ID}
\hat{\mM}(\hat{\vq}) \ddot{\vq}_d
+ \hat{\mC}(\hat{\vq},\hat{\dot{\vq}})
+ \hat{\mG}(\hat{\vq})
+ \hat{\mF}(\hat{\vq},\hat{\dot{\vq}})
+ \hat{\mJ}_w^{\tr}(\hat{\vq}) \hat{\vf}_w
\end{equation}
If the models and estimated state (position and velocity) are perfect, we get
perfect cancellation of most of the dynamics:
\begin{equation}
\mM(\vq) (\ddot{\vq} - \ddot{\vq}_d) = \mJ_o^{\tr}(\vq) \vf_o
\end{equation}
If $\ddot{\vq}_d$ is set to zero, the operator directly drives the acceleration
of the exoskeleton with an effective inertia of $(\mM^{-1}(\vq) \mJ_o^{\tr}(\vq))^{-1}$
\begin{equation}
\ddot{\vq} = \mM^{-1}(\vq) \mJ_o^{\tr}(\vq) \vf_o
\end{equation}
$\ddot{\vq}_d$ can be used to drive the exoskeleton along trajectories, and
the operator can directly modify the trajectory.

Simplifying Equation X by assuming the operator directly applies torques at the
exoskeleton joints we get:
\begin{equation}
\mM (\vq) \ddot{\vq} = \vtau_o
\end{equation}
The operator sees the true inertia of the exoskeleton, but no gravitational
loads and no other dynamics (Coriolis, centripetal, or frictional).
Note that we cannot change the apparent inertia of the exoskeleton without
some form of acceleration or force feedback.

Online inverse dynamics control is a generalization of active gravity compensation to include
inertial, Coriolis, and centripetal forces (forces due to acceleration and
velocity).
If frictional forces are modeled they can also be canceled.
Online inverse dynamics can be added to open loop behavior execution
and many other types of control such as operator force
feedback control, impedance control, and admittance control.
Computed torque control and feedback linearization are forms of online
inverse dynamics.

A key question is where does desired acceleration come from?
\begin{verbatim}
  Operator force control.
  Receding Horizon Control of optimal trajectory of 
    idealized LIPM+flywheel model (Atkeson)
  intent prediction (see above)
  neuromuscular model (Geyer, Wang, Van de Panne)
  hybrid: Optimal control + neuromuscular model (Atkeson NSF Proposal)
    - NMM provides bias through changes to optimization criterion
    - NMM provides hard constraints
    - NMM replaces components of model-based optimization scheme
\end{verbatim}

\subsection{Impedance Control}

What is impedance?
\url{https://en.wikipedia.org/wiki/Mechanical_impedance}
Stiffness, damping, and mass are components of an impedance, as
they map kinematic variables (position, velocity, and acceleration)
to forces and torques.

The goal here is to make the exoskeleton imitate a desired linear dynamic system
from the point of view of the operator.
A future white paper will discuss how to make the exoskeleton 
imitate a desired linear dynamic system 
from the point of view of external perturbations.
In the case where there is no force sensing between the operator and the
exoskeleton, the exoskeleton controller maps from exoskeleton configuration
and velocity to desired joint/actuator/synergy forces.
In this case the actuators need to generate forces and torques rather than
positions or angles, or linear or angular velocities.

Here we make the exoskeleton appear as a locally linear impedance in joint
coordinates to the operator.
We can also make the exoskeleton appear as a locally linear impedance to the
operator in some other coordinate system, such as Cartesian coordinates.
A different goal is to make the exoskeleton appear as a locally linear impedance to
the outside world in some coordinate system.

To make the exoskeleton appear as a locally linear impedance in joint
coordinates to the operator, we add position and velocity feedback to $\vtau$:
\begin{equation}
\vtau = \vtau_{ID} - \mK_{\vq} ( \hat{\vq} - \vq_d ) - \mK_{\dot{\vq}} \hat{\dot{\vq}}
\end{equation}
so if the dynamic models and state estimation are perfect we get this impedance:
\begin{equation}
\mM \ddot{\vq} - \mK_{\vq} ( \vq - \vq_d ) - \mK_{\dot{\vq}} \dot{\vq} = \vtau_o
\end{equation}
and the operator sees the desirecd impedance.

Although impedance control is extensively utilized in rehabilitation
robotics [18], limited number of studies has been focused on
this method for power augmentation [19, 20].
\begin{verbatim}
[18] R. Riener, L. Lunenburger, S. Jezernik, M. Anderschitz, G. Colombo,
and V. Dietz, "Patient-cooperative strategies for robot-aided treadmill
training: first experimental results," Neural Systems and Rehabilitation
Engineering, IEEE Transactions on, vol. 13, pp. 380-394, 2005.
[19] B.-K. Lee, H.-D. Lee, J.-y. Lee, K. Shin, J.-S. Han, and C.-S. Han,
"Development of dynamic model-based controller for upper limb
exoskeleton robot," in Robotics and Automation (ICRA), 2012 IEEE
International Conference on, 2012, pp. 3173-3178.
[20] W. Yu, J. Rosen, and X. Li, "PID admittance control for an upper limb
exoskeleton," in American Control Conference (ACC), 2011, 2011, pp.
1124-1129.
\end{verbatim}

\subsection{Force feedback and get out of the way control (inverse dynamics version)}

The goal here is to use force sensing between the operator and the
exoskeleton to ultimately generate exoskeleton velocities or angular velocities,
imitating a desired linear dynamic system.
This type of control is often referred to as admittance control as the
exoskeleton maps contact forces into joint motion.
Inverse dynamics control
can be used to improve this type of control performance.

What is admittance?
\url{https://en.wikipedia.org/wiki/Admittance}
Compliance, inverse damping, inverse mass are components of an admittance,
as they map force and torques to kinematic variables (position, velocity, and
acceleration). 

We can implement force control 
if measurements of the contact forces 
the exoskeleton are available.
The exoskeleton moves as to create as little force as possible 
assuming the actuators are torque sources.
\begin{equation}
\vtau = \vtau_{ID} - \mK_{\vf} \mJ^{\tr} ( \hat{\vf} - \vf_d )
\end{equation}
$\mK_{\vf}$ is a gain matrix, $\mJ$ is the appropriate Jacobian matrix for
force $\vf$
Note that this does not require inverting the Jacobian matrix.
We will see that the position control version does invert the Jacobian matrix.

Applying this to force between the operator and the exoskeleton, and assuming
that the operator directly applies exoskeleton joint torques,
\begin{equation}
\vtau = \vtau_{ID} - \mK_{\vtau_o} ( \hat{\vtau}_o - {\vtau_o}_d )
\end{equation}
So if our dynamic models and state estimation are perfect, we get:
\begin{equation}
\ddot{\vq} = - \mM^{-1} (\vq) \mK_{\vtau_o} ( \hat{\vtau}_o - {\vtau_o}_d )
\end{equation}

One can add force damping terms, which sometimes help:
\begin{equation}
\vtau = \vtau_{ID} - \mK_{\vtau_o} ( \hat{\vtau}_o - {\vtau_o}_d )
- \mK_{\dot{\vtau}_o} \hat{\dot{\vtau}}_o
\end{equation}

Note that impedance control can also be implemented using force control
by setting:
\begin{equation}
{\vtau_o}_d = \mK_{\vq}( \hat{\vq} - \vq_d ) + \mK_{\dot{\vq}} \hat{\dot{\vq}}
\end{equation}
in addition to directly including the stiffness and damping terms in
equation YYY. 

\subsection{Operator-Exoskeleton Force Control using position controlled actuation}

At this point our dynamics formulation is no longer appropriate, since we
are assuming the actuators are now position rather than torque sources.

\subsubsection{Get out of the way control}

The exoskeleton moves as to create as little force as possible between the
operator and the exoskeleton. The actuators in this case may be position sources or
forces sources with a high servo gains,
but ultimately operator-exoskeleton forces are mapped
to exoskeleton velocities or angular velocities.

\begin{equation}
\dot{\vq}_d = \mJ^{-1} (\mK_1 \vf)
\end{equation}
where $\dot{\vq}_d$ are the commanded exoskeleton joint velocities,
$\mJ$ is an appropriate Jacobian matrix, $\mK_1$ is a gain matrix,
$\vf$ is the force vector to be controlled.
[06990981]

Inverting a Jacobian matrix is problematic when the Jacobian matrices are
near singular.
Using a fixed or gain scheduled gain matrix may make more sense.
\begin{equation}
\dot{\vq}_d = \mK_3 \vf
\end{equation}
Simplifying Equation Z by assuming the forces to be controlled
are directly applied to the exoskeleton joints, and thus the Jacobian matrix
is the identity matrix, gives us a similar equation:
\begin{equation}
\dot{\vq}_d = \mK_1 \vtau_o
\end{equation}
and we avoid inverting a Jacobian and problems with singularities.

\subsubsection{Operator force feedback}

We introduce a desired force to allow for more complex force control.
\begin{equation}
\dot{\vq}_d = \mJ^{-1} (\mK_1 (\vf - \vf_d))
\end{equation}

The exoskeleton could move as to provide a scaled version of exoskeleton-world
contact forces, or some other (usually simple) mapping. 
\begin{equation}
\vf_d = \alpha \mJ_{ow} \vf_w
\end{equation}

\subsubsection{Integral admittance control}

{\it [Exoskeletons] have the implicit property
of causing a virtual modification of the dynamic response of
the human limb. We use this property of the exoskeletons
action to formulate a unified control design framework called
Integral Admittance [torque to angle] Shaping, which designs exoskeleton con-
trollers capable of producing the desired dynamic response
for the assisted limb. In this framework, a virtual increase
in the admittance of the limb is produced by coupling it
to an exoskeleton that exhibits active behavior. Specifically,
our framework shapes the magnitude profile of the integral
admittance (i.e. torque-to-angle relationship) of the coupled
human-exoskeleton system, such that the desired assistance is
achieved. This framework also ensures that the coupled stability
and passivity are guaranteed.}~\cite{Nagarajan_etal_2015}

{\it ... the impedance of the coupled human-exoskeleton
system needs to be reduced below that of the unassisted
human limb. This implies that the exoskeleton needs to
cancel its own impedance first and then compensate for
at least a part of the human limb’s impedance. Therefore,
the desired exoskeleton behavior must be that of a negative
impedance ... Consequently,
the feedback gains will all be
positive. In other words, the exoskeleton controller uses
positive feedback, and hence the exoskeleton exhibits active
behavior, which is capable of performing net positive work
on the limb. ...
However, positive feedback naturally raises the question of
stability, and so we now explain how coupled stability can be
achieved. Although the exoskeleton exhibits active behavior,
which can be potentially destabilizing, the controller can be
designed such that the coupled human-exoskeleton system
is stable and passive. ... 
}~\cite{Nagarajan_etal_2015}

{\it
... Inertia compensation is more complex ...
It can be shown that using only positive acceleration
feedback ..., the gain margin of the coupled system
reduces to the moment of inertia of the exoskeleton, which
implies that the exoskeleton controller ... can at the most
compensate for the exoskeleton’s own moment of inertia
before going unstable. This implies that the moment of
inertia of the coupled human-exoskeleton system cannot be
reduced below that of the unassisted human limb, without
compromising coupled stability. However, using low-pass
filtered acceleration feedback, it can be shown that inertia 
reduction can be achieved. ...
This work uses filtered acceleration
feedback with a second-order low-pass Butterworth filter. ...
}~\cite{Nagarajan_etal_2015}

This is a form of frequency domain virtual model control.

We are skeptical that acceleration feedback (even low pass filtered acceleration
estimated by a Kalman filter) will
work well on real systems. Impacts, shock waves, and the fact that neither
the operator's body parts or the exoskeletons parts are rigid bodies and the
joints are not well defined suggest this approach would be fragile to modeling
error and unmodeled dynamics.

\begin{verbatim}
Kalman filter:
P. Canet, “Kalman filter estimation of angular velocity and accelera-
tion: On-line implementation,” McGill University, Montr ́eal, Canada,
Tech. Rep. TR-CIM-94-15, Nov. 1994.
\end{verbatim}

\subsection{Virtual model control}

The goal here is to make the exoskeleton imitate a desired (usually nonlinear)
dynamic system,
For running it could be a pogo stick or trampoline, for example.

There are several possible versions of this type of control:
\begin{enumerate}
\item
Map from operator-exoskeleton contact forces and system state
to exoskeleton actuator or internal forces.
\item
Map from operator-exoskeleton contact forces and system state
to exoskeleton acceleration.
\item
Map from operator-exoskeleton contact forces and system state
to exoskeleton-world contact forces.
\item
Map from exoskeleton-world contact forces and system state to exoskeleton actuator or internal forces.
\item
Map from exoskeleton-world contact forces and system state to exoskeleton acceleration.
\item
Some combination of the above.
\end{enumerate}

\begin{verbatim}
virtual model approach: hopper, compass gait (J. Pratt)
estimate trajectory (jumping) vs. program behavior (trampoline, hopper,
   compass gait)
- J' control + behavior
- Inverse dynamics + behavior
\end{verbatim}

\subsection{Task Specific Control}

Task specific control can involve switching low level control modes, or abstracting
the behavior of the exoskeleton with a hierarchy:
\begin{enumerate}
\item
Task specific control: Generate desired motions and contact forces between
the operator and the exoskeleton, and the exoskeleton and the world.
\item
Sometimes there are intermediate levels of control.
\item
Exoskeleton control: Control the exoskeleton to generate the desired motions and
desired contact forces, with the desired impedance or admittance.
\end{enumerate}

\subsection{Sensitivity Amplification Control}

Sensitivity Amplification Control (SAC) is a dynamic cancellation technique (a.k.a.
inverse dynamics, computed torque, or feedback linearization).

BLEEX:
Sensitivity amplification control (SAC) was
developed by Kazerooni [14]. In this method, the interaction
forces with the human are regarded as disturbances. The control
system has high sensitivity to disturbances and thus has fast
motion response to the interaction forces. The main drawback
of SAC is unwanted responses to any disturbance regardless of
being intentional or non-intentional disturbance. In addition,
highly accurate dynamic modelling is required in this
method.
\begin{verbatim}
H. Kazerooni, J.-L. Racine, L. Huang, and R. Steger, "On the control of
the berkeley lower extremity exoskeleton (BLEEX)," in Robotics and
Automation, 2005. ICRA 2005. Proceedings of the 2005 IEEE
International Conference on, 2005, pp. 4353-4360.
\end{verbatim}

\subsection{Dual Control Appproach}

{\it The robot utilized the dual-mode control scheme, which is
comprised of the active control for the stance phase and the
passive control (using bypass valves) for the swing phase, to achieve high walking
speed in the swing phase while supporting heavy loads in
the stance phase. To reduce the sudden change of the torque
command at the transition from the swing phase to the stance
phase, a smoothing method is adopted. We also implemented
a pre-transition method to take a foot off quickly for fast
walking by predicting the change from the swing to the stance
in advance.}~\cite{IEEE07222598}

Bypass valves implement what Sarcos calls ``dangle''.

stance: virtual joint torque control method Kazerooni et al.(2005)
When contact location between the wearer and the
exoskeleton is not fixed and difficult to estimate, this method
has been shown to be an effective method to generate the
locomotion for an exoskeleton robot.[3],[4],[21]
\begin{verbatim}
H. Kazerooni , Z. Racine , L. Huang and R. Steger ”On the control of
the Berkeley lower extremity exoskeleton (BLEEX),” Proc. IEEE Int.
Conf. Robot. Autom., pp. 4364-4371, 2005.
Racine Jean-Louis Charles ”Control of a Lower Extrmity Exoskeleton
for Human Performance Amplification,” University of California,2003.
582
Xiuxia Yang, Hongchao Zhao, Yi Zhang, Xiaowei Liu ”Carrying
Lower Extreme Exoskeleton Rapid Terminal Sliding-Mode Robust
Control,” Journal of computers, vol.9, No.1,202-208. 2014.
\end{verbatim}

to reduce
sudden changes(command jump) at the phase transitions, a
smoothing method [5],[19] is introduced and a pre-transition
method is used to solve the swing delay due to the internal
pressure(approx 5 bar).
\begin{verbatim}
H. Kazerooni , Ryan Steger, Lihua Huang ”Hybrid Control of the
Berkeley Lower Extremity Exoskeleton (BLEEX),” Int. Journal of
Robotics Research, pp. 561-573, 2006.
oonbum Bae, Kyoungchul Kong, Masayoshi Tomizuka ”Gait Phase-
Based Control for a Rotary Series Elastic Actuator Assisting the knee
\end{verbatim}

Transition Control

1) Smoothing Method: During transitions of the gait
phase, discontinuity of the control command torque are
occurred by the different condition for the fixed coordinate
which is on the backpack at the swing phase or the foot at
the stance phase. In this paper, to reduce this sudden change
due to gait phase changes, a smoothing method is proposed
as shown in Fig. 7. An exponential function is considered
as the weighting function for smoothing as shown in (6).
In this case, the weighting is small at the initial stage but
it exponentially converges to one for supporting the load
quickly.

2) Pre-transition Method: The pre-transition method is
that the passive mode is executed in the pre-swing phase
prior to toe off. This dramatically reduces the moving

\subsection{Ground Reaction Force Control}

The ground reaction
forces (GRF) magnitude and direction is used to command the
actuators. In some researches the GRF sensors are used
together with other sensors in the control architectures [17],
while in other researches the control system is based on the
GRF merely [RoboKnee, Honda].
\begin{verbatim}
17: K. Suzuki, G. Mito, H. Kawamoto, Y. Hasegawa, and Y. Sankai,
"Intention-based walking support for paraplegia patients with Robot
Suit HAL," Advanced Robotics, vol. 21, pp. 1441-1469, 2007
\end{verbatim}

\subsection{Survey literature and current exoskeleton systems for other approaches}

\subsection{How handle hand and other non-foot contact forces?}

Admittance control?

Map external contact forces to internal contact forces (operator-exoskeleton).

\section{Control Architectures}

A generic hierarchical control architecture~\cite{IEEE06907051}:

{\bf Lowest level Control:} present interface to actuators and
high bandwidth feedback control, degrees of freedom,
or synergies of either:
\begin{itemize}
\item
Position control.
\item
Pure force or torque control.
\item
Impedance control: specify desired position (for spring), and generate
stiffness, damping, and potentially a modified mass/iinertia/moment of inertia.
Modifying inertia is very difficult.
Or (almost equivalently),
specify a desired force and change the force according to position, velocity,
and acceleration.
\end{itemize} 
One important synergy is coordinated control of the center of mass location and
velocity for balance.
PID or fancier feedback control laws are implemented at this level.
Gravity, friction, actuator dynamics, and other feedforward disturbance
compensation is implemented at this level, as well as any inverse dynamics
or other types of feedforward reference tracking control.
Disturbance observers for model errors and contact/wind forces
are implemented at this level.
Typically this is done by a high rate global servo~\cite{Atlas-robot}
or local limb or joint-level servos~\cite{Sarcos-robot}.

{\bf ``Phoneme'' Control:} mid-level controllers
for relatively constant phases of behavior.
Often involves position or velocity targets (goto or
stay-at X), trajectory references, ...
Example from walking: left-stance, double-support-1, right-stance, double-support-2, ...;
Example from running: left-foot-down, flight-1, right-foot-down, flight-2, ...;
Example from horizontal jumping: hjump-pushoff, hjump-flight, hjump-prepare-landing, hjump-impact, hjump-balance, ...;
Finer resolution may be needed. For example a walking stance period may
be broken up into heel-strike, first-half, second-half, pushoff, and toe-off.
The first and second halves of stance are divided by when the center of mass
passes over the ankle.
Local inverse kinematics (finding small joint motions for small target changes)
is performed at this level.

{\bf ``Word'' Control:} mid-level controllers for sequences of ``phoneme'' behavioral
units.
Often involves finite state machines, timers, if then condition testing, gain
scheduling, ...
Examples: walk-to-location( x=7.8m, y=2.3m ), run-to-location( speed=3.7m/s ),
horizontal-jump( height= 2m, length= 5m ).
Global inverse kinematics (selection of solution branch) is performed at this level.

{\bf Highest level control} aka {\bf Behavior Selection:} 
Answer the question: ``What do I do next?''
Select behaviors and behavioral parameters (targets,
speeds, durations).
Often involves finite state machines. timers, if then condition testing, ...

{\bf Error handling:} Errors must be detected and handled by behavior switching
at all of these levels.

An alternative behavioral architecture does not have explicit levels but instead
has a ``soup'' of behaviors that compete or cooperate for control.

Another alternative behavioral architecture does not have explicit separate behaviors,
but one big function (produced by deep learning?).

\section{Symbiosis and Autonomy, Handling Errors, and Superhuman Reflexes}

A later white paper will discuss
possible combinations of response to user and exoskeleton
autonomy including balance control; response to trips, slips,
stumbles, and fumbles; task assistance and guidance (guide operator to
doorknob, button, or light switch); superhuman response to external
perturbations such as projectiles and explosions; and autonomous
execution (operator/exoskeleton symbiosis with multi-tasking)?

\section{Do we need physical contact?}

Current exoskeletons use physical contact with the operator to control
the exoskeleton. Can we make an exoskeleton that does not touch or
minimally touches the operator, and in which there is no power
transfer between the operator and the exoskeleton? These types of
systems might allow an operator to behave more normally, have a
greater range of movement, be less tiring, and for non-enclosing
systems, allow the operator to move independently of the system (such
as aim and fire standard weapons or dive for cover) for additional
operator performance and safety.
\begin{enumerate}
\item
A "fat suit" or "sumo suit" could fully enclose the operator, but
use distance sensing to move with the operator.
\item
A "shield" could enclose the front of an operator (or whatever
portion of the operator is threatened) and move with the operator,
\item
A "human shield" (actually a machine shield) could move ahead of
an operator, mimicking the operators movement.
\item
An "angel" could walk, jump, or fly between an operator and potential
hostile sites, and deflect or disable projectiles.
\end{enumerate}

\section{One Time Use?}

What happens if the suit is "one time use"? For example, ablative armor,
armor that reacts to incoming projectiles before they hit,
or explosive shielding could be used. Could the armor weight be decreased?
The expensive parts of the suit could be re-usable, and even autonomously
return to safe areas.

\section{Conclusions and Recommendations}

Conclusions and Recommendations to be written.

\bibliographystyle{plain}
\bibliography{exo}

\newpage

\section{Appendix: Our List of Exoskeletons}

Please tell us what is missing.

\subsection{Surveys}

\begin{verbatim}
Surveys:
http://exoskeletonreport.com/
http://robohub.org/tag/exoskeleton/
http://robohub.org/exoskeletons-new-and-older/
http://neurogadget.com/tag/exoskeleton
http://www.engadget.com/tag/exoskeleton/
http://spectrum.ieee.org/robotics/medical-robots/exoskeletons-around-the-world
http://spectrum.ieee.org/robotics/medical-robots/exoskeletons-are-on-the-march
http://spectrum.ieee.org/biomedical/bionics/the-rise-of-the-body-bots
http://www.extremetech.com/electronics/139633-will-we-ever-have-iron-man-exoskeletons
http://futurewarstories.blogspot.com/2014/08/fws-topics-combat-exoskeletons.html
https://prezi.com/jygcvfxiesnm/a-brief-introduction-to-biomechanical-exoskeletons/
http://news.discovery.com/tech/robotics/exoskeleton-robots-top-5.htm
http://blog.equipoisinc.com/new-technologies-emerging-aerospace-defense-manufacturing/
http://nextbigfuture.com/2015/03/lower-body-exoskeleton-audi-chairless.html

H. Lee, W. Kim, J. Han, and C. Han, ”The technical trend of the
exoskeleton robot system for human power assistance, ” International
Journal of Precision Engineering and Manufacturing, vol. 13, pp. 1491-
1497, 2012.

E. Guizzo and H. Goldstein, "The rise of the body bots [robotic
exoskeletons]," Spectrum, IEEE, vol. 42, pp. 50-56, 2005.

A. M. Dollar, H. Herr, Lower extremity exoskeletons and active
orthoses: challenges and state-of-the-art, IEEE
Transactions on Robotics: 24 (2008) 144-158.
(surveys exoskeletons)
http://excedrin.media.mit.edu/wp-content/uploads/sites/3/2013/07/Dollar-2008_Lower-Extremity-Exoskeletons-and-Active-Orthoses-Challenges-and-State-of-the-Art.pdf

Simone Cinquemani Federico Casolo and Matteo Cocetta. On
active lower limb exoskeletons actuators. ISMA 08, 2008
\end{verbatim}

\subsection{Exoskeletons}

\begin{verbatim}
This is on the low end of the ballpark: 1.1m/s (4km/hr), 99lb (45kg) payload
IEEE 07222598
2015 IEEE International Conference on Advanced Intelligent Mechatronics (AIM)
July 7-11, 2015. Busan, Korea
Locomotion Control Strategy of Hydraulic Lower Extremity
Exoskeleton Robot
Hongchul Kim, Changhoon Seo, Young June Shin, Jung Kim, and Youn Sik Kang

BLEEX: Berkeley Lower Extremity Exoskeleton

Ekso Bionics
http://www.bloomberg.com/bw/articles/2013-03-28/businesses-bet-on-iron-man-like-exoskeletons

Sarcos XOS/WEAR
http://www.army-technology.com/features/featuremilitary-exoskeletons-uncovered-ironman-suits-a-concrete-possibility
http://www.robaid.com/bionics/raytheons-second-generation-exoskeleton-xos-2.htm
http://www.raytheon.com/newsroom/technology/rtn08_exoskeleton/
http://techcrunch.com/2010/09/27/sarcos-xos2-the-real-life-iron-man/
S. Jacobsen, ”On the development of XOS, a powerful exoskeletal
robot,” IEEEIRSJ IROS, Plenary Talk, 2007.

Lockheed Martin HULC, FORTIS
http://www.lockheedmartin.com/us/news/features/2014/mfc-103114-relief-daily-grind-industrial-exoskeletons-work.html
http://www.lockheedmartin.com/us/products/exoskeleton/FORTIS.html
http://www.dailytech.com/From+HULC+to+FORTIS+the+Evolution+of+Lockheed+Martins+Incredible+Exosuit/article36421.htm
http://www.wired.com/2014/09/navys-exoskeleton-could-make-workers-20-times-more-productive/
http://robrady.com/design-project/lockheed-martin-fortis-human-powered-exoskeleton
http://www.cnn.com/2013/05/22/tech/innovation/exoskeleton-robot-suit/

DKFI CAPIO and Exoskeleton Active (VI-Bot)
http://robotik.dfki-bremen.de/en/research/robot-systems/exoskeleton-passive.html
http://robotik.dfki-bremen.de/en/research/robot-systems/exoskeleton-active.html

DAEWOO
http://mashable.com/2014/08/05/daewoo-exoskeleton/#xz8IUKO3bmk_
http://www.geek.com/science/daewoo-dock-workers-now-using-strength-enhancing-exo-skeletons-1601290/

Vanderbilt
http://www.dailytech.com/Iron+Man+New+Robotic+Exoskeleton+Helps+Paraplegics+Walk/article29079.htm
http://research.vuse.vanderbilt.edu/cim/research_orthosis.html

Vanderbilt/Parker-Hannafin
http://www.indego.com/indego/en/home
https://www.youtube.com/watch?v=-bYYmZNxaZk
https://en.wikipedia.org/wiki/Vanderbilt_exoskeleton
http://www.roboticstrends.com/article/parker_hannifin_indego_exoskeleton
http://www.crainscleveland.com/article/20141112/BLOGS03/141119915/parker-hannifin-accepts-the-challenge-of-commercializing-the-indego--41951672-812546791-1442419256=:6352--

Activelink/Panasonic
http://pinktentacle.com/2009/09/power-loader-exoskeleton-suit/
http://www.extremetech.com/extreme/173463-worlds-first-affordable-powered-exoskeleton-is-almost-here-prepare-for-mech-wars
http://www.dealstreetasia.com/stories/panasonic-to-release-commercial-exoskeleton-to-market-in-4q-2015-9269/

Cyberdyne: HAL 2000 up to at least HAL 5
http://www.computerweekly.com/photostory/2240108388/Photos-Cyberdyne-Hal-robotic-exoskeleton-to-help-paralyzed/7/Japan-beats-the-US-to-it-Cyberdyne-Hal-robotic-exoskeleton-to-help-paralyzed
http://spectrum.ieee.org/robotics/medical-robots/exoskeletons-are-on-the-march
http://www.dailymail.co.uk/sciencetech/article-2384930/Robotic-exoskeleton-help-rehabilitate-disabled-people-passes-safety-tests--paving-way-sale-UK.html
H. Kawamoto, S. Lee, S. Kanbe, and Y. Sankai, "Power assist method
for HAL-3 using EMG-based feedback controller," in Systems, Man
and Cybernetics, 2003. IEEE International Conference on, 2003, pp.
1648-1653.

Noonee
http://www.extremetech.com/extreme/188417-rejoice-commuters-and-workers-the-chairless-chair-exoskeleton-lets-you-sit-down-anywhere-anytime

Kanagawa Institute of Technology
pneumatic exoskeleton developed at the Kanagawa Institute of Technology,
in Atsugi, Japan, allows Akiko Michihisa, a fitness trainer, ...
http://www.ubergizmo.com/2007/10/air-pressure-exoskeleton/

Monash/Chen
http://www.dailymail.co.uk/sciencetech/article-2639939/Super-fireman-The-hi-tech-exoskeleton-firefighters-superhuman-abilities.html
http://news.discovery.com/tech/robotics/firefighter-exoskeleton-to-the-rescue-140521.htm

Sagawa Electronics
https://www.youtube.com/watch?v=tYgHdxmCcAI
http://www.ohgizmo.com/2013/07/11/mk3-exoskeleton-suit-promises-to-make-schoolgirls-better-taller-stronger/
http://www.bitrebels.com/technology/powered-jacket-exoskeleton-japan/

Micromega
http://www.micromega-dynamics.com/esa-projects/exoskeleton.html

UCLA
http://bionics.seas.ucla.edu/research/exoskeleton_device_3.html
http://bionics.seas.ucla.edu/research/exoskeleton_device_2.html

RoboKnee
J. E. Pratt, B. T. Krupp, C. J. Morse, and S. H. Collins, "The
RoboKnee: an exoskeleton for enhancing strength and endurance
during walking," in Robotics and Automation, 2004. Proceedings.
ICRA'04. 2004 IEEE International Conference on, 2004, pp. 2430-
2435.

NASA/IHMC
http://www.nasa.gov/offices/oct/home/feature_exoskeleton.html#.VfHU72wVhBc
http://www.dailymail.co.uk/sciencetech/article-2216748/Nasa-developing-exoskeleton-help-astronauts-exercise-zero-gravity-help-disabled-people-walk-Earth.html
http://robots.ihmc.us/x1-mina-exoskeleton/

P.D. Neuhaus, et al., “Design and evaluation of Mina: A robotic
orthosis for paraple ics,” in Rehabilitation Robotics (ICORR), 2011
IEEE International Conference on. 2011. IEEE.

DARPA Warrior Web program
http://gajitz.com/darpas-exoskeleton-can-help-soldiers-run-4-minute-miles/
http://www.dailymail.co.uk/sciencetech/article-2348174/Robotic-exoskeleton--soft-light-pulls-like-pair-trousers--day-soldiers-superhuman-strength.html
http://www.theengineer.co.uk/in-depth/the-big-story/power-dressing-why-its-exoskeleton-time/1019633.articlehttp://www.theengineer.co.uk/in-depth/the-big-story/power-dressing-why-its-exoskeleton-time/1019633.article
http://www.engadget.com/2014/09/12/darpa-harvard-soft-exosuit/

ESA
http://www.esa.int/spaceinimages/Images/2014/06/Ready_to_wear_exoskeleton
http://www.esa.int/spaceinimages/Images/2014/06/Exoskeleton
http://www.sciencefocus.com/blog/exoskeleton-turning-man-machine

3D Systems
http://www.dezeen.com/2014/03/05/3d-printed-exoskeleton-helps-paralysed-users-walk/

U.Penn
http://www.popsci.com/article/technology/invention-awards-2014-powerful-portable-and-affordable-robotic-exoskeleton
https://www.youtube.com/watch?v=kAyI1YrWThQ

MIT Herr:
http://biomech.media.mit.edu/portfolio_page/load-bearing-exoskeleton-for-augmentation-of-human-running/

MIT exo
http://biomech.media.mit.edu/portfolio_page/load-bearing-exoskeleton-for-walking/
http://www.technovelgy.com/ct/Science-Fiction-News.asp?NewsNum=1205
C. J. Walsh, K. Endo, and H. Herr, "A quasi-passive leg exoskeleton
for load-carrying augmentation," International Journal of Humanoid
Robotics, vol. 4, pp. 487-506, 2007.

Honda (Stride Management Assist (SMA))
http://www.slashgear.com/honda-demo-walking-assist-exoskeleton-for-elderly-disabled-2211295/
http://singularityhub.com/2010/09/13/hondas-exoskeletons-help-you-walk-like-asimo-video/
Honda Walking Assist Device
Y. Ikeuchi, J. Ashihara, Y. Hiki, H. Kudoh, and T. Noda, "Walking
assist device with bodyweight support system," in Intelligent Robots
and Systems, 2009. IROS 2009. IEEE/RSJ International Conference on,
2009, pp. 4073-4079.

REX exoskeleton
http://www.rexbionics.com/
http://www.gizmag.com/rex-robotic-exoskeleton/15736/pictures
http://www.stripes.com/news/exoskeleton-could-benefit-troops-with-spinal-cord-injuries-1.113657
http://www.funis2cool.com/cool/rex-the-robotic-exoskeleton-could-change-the-lives-of-disabled-people.html

Waterloo
http://www.eng.uwaterloo.ca/~jpiorkow/fydp/

Soft pneumatic exo
http://cwwang.com/2008/04/08/soft-pneumatic-exoskeleton/

ReWalk
http://www.businessinsider.com/human-exoskeleton-approved-by-fda-2014-6

Revision Military PROWLER Human Augmentation System (HAS) Exoskeleton
https://www.youtube.com/watch?v=Zq4amM9u-6o

Yasakawa
http://www.crunchwear.com/yaskawa-electrics-ankle-exoskeleton-gives-strength/

http://www.engadget.com/2015/08/20/japans-top-oil-company-is-building-an-aliens-power-loader/

http://robohub.org/clinical-trials-begin-for-russias-first-medical-exoskeleton/

https://en.wikipedia.org/wiki/Z_series_(space_suits)

https://en.wikipedia.org/wiki/Extravehicular_Mobility_Unit

Passive:
http://techcrunch.com/2011/03/23/japanese-engineers-create-human-powered-exoskeleton-suit-video/
http://www.jneuroengrehab.com/content/6/1/21/figure/F2?highres=y

SUEFUL-7: A 7DOF upper-limb exoskeleton robot with muscle-model-oriented
EMG-based control
RARC Gopura, K Kiguchi, Y Li - Intelligent Robots and Systems …, 2009 -
ieeexplore.ieee.org

EU BALANCE program
http://balance-fp7.eu/
points us to European work
http://cordis.europa.eu/project/rcn/106854_en.html

MINDWALKER
key feature is actuation in lateral plane
http://www.3me.tudelft.nl/en/about-the-faculty/departments/biomechanical-engineering/research/dbl-delft-biorobotics-lab/exoskeleton/

https://mindwalker-project.eu/

http://www.utwente.nl/ctw/bw/research/projects/MINDWALKER/
http://tnsre.embs.org/2015/03/27/design-control-mindwalker-exoskeleton/

Each of the powered joints has a series elastic
actuator, which can deliver 100 Nm torque and 1 kW power -> 10rad/s
IEEE TRANSACTIONS ON NEURAL SYSTEMS AND REHABILITATION ENGINEERING, VOL. 23, NO. 2, MARCH 2015 277-
Design and Control of the MINDWALKER
Exoskeleton
Shiqian Wang, Letian Wang, Cory Meijneke, Edwin van Asseldonk, Thomas Hoellinger, Guy Cheron,
Yuri Ivanenko, Valentina La Scaleia, Francesca Sylos-Labini, Marco Molinari, Federica Tamburella,
Iolanda Pisotta, Freygardur Thorsteinsson, Michel Ilzkovitz, Jeremi Gancet, Yashodhan Nevatia,
Ralf Hauffe, Frank Zanow, and Herman van der Kooij


HEIKA
https://www.heika-research.de/26_166.php
HEiKA-EXO - Optimization-based development and control of an exoskeleton
for medical applications

http://www.beyondhd.tv/exoskeleton/

https://www.youtube.com/watch?t=3&v=myNRcDIgfjw
www.livescience.com/51940-mind-controlled-exoskeleton-robot.html
http://www.ibtimes.co.uk/scientists-develop-mind-controlled-robotic-exoskeleton-that-uses-leds-help-paraplegics-walk-1516834

http://csl-ep.mech.ntua.gr/index.php/research/telerobotics-and-servohydraulic-simulators/emg-exoskeleton-control

http://www.sarna.net/news/maxfas-exoskeleton-human-fire-control-supports-against-fatigue/
MAXFAS Exoskeleton: Human Fire Control Supports against Fatigue

Historical
Cornell Aeronautical Labs with funding from the US Air Force. This
resulted in the 1961-1962 Man-Amplifier suit that was more of an working
model than a working prototype.
The Cornell exo-suit was a mockup, and never included motors, servos, or
power.

GE's Hardiman I industrial exoskeleton project from 1965-1971 was the
first working exo-suit.
While a full mockup model was constructed, only pieces of Hardiman I
exo-suit worked, namely the claw-like arms. While the arms worked, the
leg-based frame work were an issue, especially with locomotion. By 1971,
GE canceled the project.
http://www.neatorama.com/2008/05/04/building-your-own-functional-iron-man-suit/

the best of the 1980's exo-suits was the 1988 NASA AX-5 Hardshell space
suit. For several decades, NASA's AMES had been working on a hadshelled
space suit system that they called the "AX" series. The apex of this
research was the 1988 AX-5.
While bulky, the AX-5 possessed 95% maneuverable of the naked human form,
and this suit was targeted for protection of astronauts in high-risk
areas. It was envisioned that the AX-5 would be used in EVAs in Earth
orbit and asteroid mining. In the end, it was determined that NASA did not
need a suit like this and alone with some remaining technical issues, the
AX hardsuits were shelved.
\end{verbatim}

\newpage

\section{Appendix: References To Be Processed}

Please suggest additional references.

\begin{verbatim}

Overviews ********************************

S.Viteckova, P.Kutilek, and M.Jirina, “Wearable Lower Limb Robotics:
A Review,” Biocybernetics and Biomedical Engineering, vol. 33, pp.
96–105, 2013.

”Lightening the Load”, U.S. Naval Research Advisory Committee
(NRAC), 2007.

Active Exoskeleton Control Systems: State of the Art
http://www.sciencedirect.com/science/article/pii/S1877705812026732

Control methodologies for upper limb exoskeleton robots
http://ieeexplore.ieee.org/xpl/login.jsp?tp=&arnumber=6427387&url=http%3A%2F%2Fieeexplore.ieee.org%2Fiel5%2F6416132%2F6426928%2F06427387.pdf%3Farnumber%3D6427387

A review of exoskeleton-type systems and their key technologies
CJ Yang, JF Zhang, Y Chen, YM Dong
http://pic.sagepub.com/content/222/8/1599.abstract

D. P. Ferris, “The exoskeletons are here,” Journal of Neuroengineering
and Rehabilitation, vol. 6, no. 17, pp. 1–3, 2009.

G. S. Sawicki and D. P. Ferris, “Mechanics and energetics of level
walking with powered ankle exoskeletons,” J. Exp. Biol., vol. 211, no.
Pt. 9, pp. 1402–1413, 2008.

http://www.bme.ncsu.edu/labs/hpl/SawickiDissertationFinal.pdf
Mechanics and energetics of walking with powered exoskeletons

Carry Payload *****************************

The first autonomously powered exoskeleton device that experimentally
demonstrated reduction in metabolic cost for human walking during
load carriage
L. M. Mooney, E. J. Rouse, and H. M. Herr, “Autonomous exoskeleton
reduces metabolic cost of human walking during load carriage,”
Journal of Neuroengineering and Rehabilitation, vol. 11, no. 80, 2014.

H. Kazerooni and R. Steger, “Berkeley lower extremity exoskeleton,”
ASME J. Dyn. Syst., Meas., Control, vol. 128, pp. 14–25, 2006.

C. J. Walsh, K. Pasch, and H. Herr, “An autonomous, underactuated
exoskeleton for load-carrying augmentation,” in Proc. IEEE/RSJ Int.
Conf. Intelligent Robots and Systems (IROS), 2006, pp. 4110–1415.

Impedance control **************************

G. A.-Ollinger, J. E. Colgate, M. A. Peshkin, and A. Goswami,
“Design of an active one-degree-of-freedom lower-limb exoskeleton
with inertia compensation,” Int. J. Robotics Research, vol. 30, no. 4,
pp. 486–499, 2011.

Learning ************************************

K. E. Gordon, C. R. Kinnaird, and D. P. Ferris, “Locomotor adaptation
to a soleus emg-controlled antagonistic exoskeleton,” J. Neurophysiol.,
vol. 109, no. 7, pp. 1804–1814, 2013.

EMG control ********************************

H. Kawamoto, S. Lee, S. Kanbe, and Y. Sankai, “Power assist method
for hal-3 using emg-based feedback controller,” in Proc. IEEE Int.
Conf. Syst., Man, Cybern., 2003, pp. 1648–1653.

S. Lee and Y. Sankai, “The natural frequency-based power assist
control for lower body with hal-3,” in Proc. IEEE Int. Conf. Sys. Man
Cyber. (ICSMC), 2003, pp. 1642–1647.

Not sorted ***********************************

master thesis design and control of a robotic exoskeleton for ...
digital.csic.es/.../Bortole_M_Design_Control_Robotic_Exoskeleton_Gait...
by M Bortole - ‎2013 - ‎Cited by 1
Control of a Robotic Exoskeleton for Gait Rehabilitation ” written by
Magdo Bortole as a requirement to obtain the Degree in the official Master
in Robotics and

Active Impedance:
A Novel Concept in Assistive Exoskeleton Control
http://www.ambarish.com/exos.html

http://gradworks.umi.com/33/86/3386362.html
Active impedance control of a lower-limb assistive exoskeleton
by Aguirre-Ollinger, Gabriel, Ph.D., NORTHWESTERN UNIVERSITY, 2009, 297
pages; 3386362

Active-impedance control of a lower-limb assistive exoskeleton
G Aguirre-Ollinger, JE Colgate… - … , 2007. ICORR 2007. …, 2007 -
ieeexplore.ieee.org

Design of an active one-degree-of-freedom lower-limb exoskeleton with
inertia compensation
G Aguirre-Ollinger, JE Colgate… - … Journal of Robotics …, 2011 -
ijr.sagepub.com

http://www.scholars.northwestern.edu/pubDetail.asp?t=pm&id=84856481679&
Inertia compensation control of a one-degree-of-freedom exoskeleton for
lower-limb assistance: Initial experiments

Power assist control for walking aid with HAL-3 based on EMG and impedance
adjustment around knee joint
S Lee, Y Sankai - Intelligent Robots and Systems, 2002. IEEE/ …, 2002 -
ieeexplore.ieee.org

http://www.intechopen.com/books/international_journal_of_advanced_robotic_syste\
ms/design-and-control-of-a-powered-hip-exoskeleton-for-walking-assistance
Design and Control of a Powered Hip Exoskeleton for Walking Assistance
By Qingcong Wu, Xingsong Wang, Fengpo Du and Xiaobo Zhang

http://www.jcomputers.us/vol9/jcp0901-29.pdf
Carrying Lower Extreme Exoskeleton Rapid
Terminal Sliding-Mode Robust Control

http://h2t.anthropomatik.kit.edu/pdf/Beil2015.pdf
Design and Control of the Lower Limb Exoskeleton KIT-EXO-1

Reducing the energy cost of human walking using an unpowered exoskeleton
http://www.nature.com/nature/journal/v522/n7555/fig_tab/nature14288_SF1.html

Control of a lower extremity exoskeleton for human performance
amplification
JLC Racine - 2003 - dl.acm.org

Locomotive control of a wearable lower exoskeleton for walking enhancement
KH Low, X Liu, CH Goh, H Yu - Journal of Vibration and Control, 2006 -
jvc.sagepub.com

Exoskeleton robots for upper-limb rehabilitation: state of the art and
future prospects
HS Lo, SQ Xie - Medical engineering & physics, 2012 - Elsevier

http://www.scielo.org.co/scielo.php?pid=S0123-30332014000100020&script=sci_arttext
Design and Implementation of a control strategy for static balance of a
lower limbs exoskeleton

Implementation of Real Time Control Algorithm for Gait ...
www.seipub.org/ABER/Download.aspx?ID=22038
Control approach follows gait trajectory using feedback sensors and
actuators for movement control. ... Exoskeleton Device; Real Time Control;
Gait Phases.

Exoskeleton control for lower-extremity assistance based on adaptive
frequency oscillators: Adaptation of muscle activation and movement
frequency
http://pih.sagepub.com/content/229/1/52.abstract

Control of a Lower Extremity Exoskeleton for Human ...
202.28.199.34/multim/3121656.pdf
The lower extremity exoskeleton is a wearable robotic device that should
... The control law makes use of dynamic models of the exoskeleton to
estimate the.

http://labrob.mty.itesm.mx/proyectos/robotic-exoskeletons
Robotic Exoskeletons

http://info.scichina.com/sciFe/EN/abstract/abstract516165.shtml
The relationship between physical human-exoskeleton interaction and
dynamic factors: using a learning approach for control applications

Online Assessment of Human-Robot Interaction for Hybrid Control of Walking
http://www.mdpi.com/1424-8220/12/1/215


\end{verbatim}

\end{document}


