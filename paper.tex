\documentclass[letterpaper,12pt,fullpage]{article}

\usepackage[left=1in,right=1in,top=1in,bottom=1in]{geometry}
\usepackage{cite}
\usepackage{graphicx}
% \usepackage[dvips]{graphicx}
% \usepackage{epsfig} % for postscript graphics files
  % \graphicspath{{../eps/}}
% \DeclareGraphicsExtensions{.eps}
\usepackage{amsmath}
\usepackage{amssymb}
%\usepackage[cmex10]{amsmath}
%\usepackage{array}
%\usepackage{mdwmath}
%\usepackage{mdwtab}
%\usepackage{eqparbox}
\usepackage[tight,footnotesize]{subfigure}
%\usepackage[caption=false]{caption}
%\usepackage[font=footnotesize]{subfig}
%\usepackage{fixltx2e}
%\usepackage{stfloats}
\usepackage{hyperref}

% correct bad hyphenation here
%\hyphenation{op-tical net-works semi-conduc-tor}

\input latex-commands

\newcommand{\T}{{\mbox {T}}}

\newcommand{\shrinkfig}{\def\baselinestretch{1.0}\small} % 0.9 okay
\newcommand{\shrink}{\def\baselinestretch{1.1}\small} % 0.97 % 0.95 okay

\begin{document}

\title{A Survey of Exoskeleton Control (Draft 1.1)}

\author{Alex Ansari, Christopher G. Atkeson, Howie Choset, and Matthew Travers\\
Carnegie Mellon University}

\maketitle

\begin{abstract}
Abstract to be written.
\end{abstract}

\section{Executive Summary}

Executive Summary to be written.

\section{What is this paper about?}

This paper surveys state of the art exoskeleton technology with an
emphasis on control. The focus is on exoskeleton technologies that allow a
highly trained and top percentile athletic 
operator to carry a payload that weighs approximately the same amount
as the operator. We envisage these types of exoskeletons to be useful
in carrying protective and safety equipment for SWAT teams, police,
firefighters, and soldiers. 

We expect each exoskeleton controller
to be used by and optimized for a single operator.
A substantial investment in capturing the operators normal behavior,
operator training and learning, and controller customization can be made.

We do not survey exoskeleton technology for human arms. 
We treat a torso and helmet of the exoskeleton as the payload,
and focus on a lower body exoskeleton to support any payloads that
are on the torso or head.
We do not cover global energy storage such as batteries or liquid fuel.

\section{What would it be like for the operator?}

\subsection{``Invisible'' exoskeleton}

Can we build an exoskeleton allows the operator to behave naturally
and exerts little or no force on the operator?

\subsection{``Natural'' exoskeleton}

Can we build an exoskeleton allows the operator to behave naturally?

\subsection{``Symbiotic'' exoskeleton}

Can we build an exoskeleton that allows the operator/exoskeleton team
to be effective? The operator needs to learn to ``fly'' the suit,
just as operators learn to operate parachutes,
wing suits, diving equipment, high and low temperature protective
suits, firefighting equipment, and high altitude flight suits.

\subsection{Physical Constraints}

1) There is a weight distribution issue. Ideally the exoskeleton should
have a center of mass at the same location as the operator. Otherwise,
operator movements can be degraded and substantial forces occur between
the operator and the exoskeleton.

This constraint can be reduced with the use of propulsive forces other
than foot contacts: rockets (solid or liquid fuel), jets, exploding armor,
ducted fan air flow, ...

A later white paper will discuss the use of propulsive forces other
than foot contacts, as well as additional actuation at the contacts (air
bags, springs, other ways to rapidly apply stored energy) for
operator safety (dodging threats, ...) and high performance (jumping, ...).

\section{Recognizing Operator Intent}

Survey current operator/exoskeleton interfaces.

The further in the future the control system can predict what the operator
is going to do, the easier it is to control the exoskeleton. One way
to think about predicting operator intent is to imagine you are in
a hand-to-hand fight and you want to predict your opponent's next move.
You can also think about playing poker. What ``tells'' does the opponent
have?
You are allowed to instrument your opponent. What sensors would you use?
How would you interpret the sensors?

\subsection{Intent time scales}

What time scales does intent operate on, and what might we use
to estimate intent on different time scales?

\begin{verbatim}
10 sec - rely on perceiving situation and thinking like your opponent.
         What are the probabilities of various attacks? Responses?
1 sec - measure set/tells to estimate probability of particular behaviors.
100 msec - EMG, muscle state, force helpful here
10 msec - EMG, muscle state, force helpful here and
         immediate sensors: joint pos,vel, IMU, contact forces
1 msec - immediate sensors: joint pos,vel, IMU, contact forces
\end{verbatim}

\subsection{Signal flow analysis}

To what extent can we use brain, neural, and muscle electrical signals
(EMG) to anticipate what the operator will do and increase performance?

\subsubsection{What signals are there?}

\begin{verbatim}
- perceive what user sees, hears, feels, smells, tastes.
- superhuman perception (RF signals, UV, IR, ultrasound, ...)
- brain signals
- spinal signals (motorneuron pools)
- motor nerve signals
- muscle signals (EMG)
- instrumenting muscles: muscle force at tendons, muscle internal pressure
- sensory nerve signals
- implanted tissue markers, other implanted sensors
- tissue imaging (ultrasound, optical)
- human/exo contact force
- contact strain, deformation rate
- exo joint/IMU sensing
- train user to emit special signals (play a videogame with their body,
  hands, eyes, neural signals, muscle signals, ...)
\end{verbatim}

A future white paper will discuss how intent prediction could be 
made easier with superhuman perception.

A future white paper will discuss how intent prediction could be 
made easier with implanted markers and devices, as well as continuous
operator imaging such as ultrasound.

\subsubsection{Signal timing?}

So for inferring the ``intent'' of a human operator of an
exoskeleton,
I would like to put a timeline together of events.
I guessed some timing, which I hope you will correct.

When I say leg EMG or leg muscle I mean
gastrocneumius or vastus lateralis (``fast'' muscles).

What are delays between:
\begin{verbatim}
visual event to leg EMG start
auditory event to leg EMG start
vestibular event to leg EMG start
foot tactile event to leg EMG start
EMG start to 10% leg muscle
EMG start to peak leg muscle
\end{verbatim}

What are your guesses for motorneuron activation to leg EMG activation?

{\b From Hartmut Geyer:}

The traveling time for neural signals is roughly 100m/s in humans.
Thus, to go from any sensory system in the head (vestibular, ocular,
auditory) to the leg muscles will take between 50ms (hip muscles) and
100ms (foot muscles). Once the signal reaches the muscle, it takes
another 5-10ms to be recognized by the surface EMG sensor (electrical
field propagation in muscle tissue is about 0.4m/s).

With that calculation:\\
vis/aud/vestib to EMG: 50-70ms for vastus and about 80-100ms for
gastroc

example: paper on vestibular reflexes reporting 60ms to
90ms:�\url{http://www.ncbi.nlm.nih.gov/pmc/articles/PMC4288134/}
��
If the motor cortex is involved with active calculations, times can
take longer (I don't know numbers for this, but would guess adding
50-100ms). I've heard that in combat soldiers, the motor cortex is
getting by-passed, and they act far more reflexively from sensory
systems directly through the lower brain. So if this is for military
applications, I'd expect to see responses mainly determined by
reflex travel times.

Foot tactile event to gastrocnemius EMG:
\begin{itemize}
\item
traveling time in neural system: 10ms foot sensors->spinal cord +
10ms spinal cord->muscle�
\item
mechanical delay foot to foot sensory cells: 5-10ms
\item
electrical delay muscle activation to EMG: 5-10ms
\item
total: 30-40ms (which is consistent with reported EMG activities
after mechanical disturbance at ankle joint with bowden cable jerk:
\url{http://jn.physiology.org/content/76/2/1112.short�reports 42ms)}
\end{itemize}

Foot tactile event to vastus EMG:\\�
- same as gastrocnemius less about 5ms for shorter travel time down
the spinal cord to the muscle.

EMG start to 10\% leg muscle force:\\
- this gets tricky: EMG has fixed delay from muscle activation due to
travel of electrical field to muscle surface (5-10ms, see top). Delay
of muscle force production follows low pass filter function (calcium
ion dynamics), often modeled as single pole filter with 10ms
characteristic time. At 10\% force, both EMG and leg force could
actually occur at the same time.�

EMG start to peak muscle force:\\
typically, (full) force production trails EMG by about 30ms in fast
muscles (electromechanical delay, due to calcium ion dynamics
traveling from sarkoplasmatic reticulum to cross bridge binding
sites, \url{http://www.ncbi.nlm.nih.gov/pubmed/527577})

alpha motoneuron activation to leg EMG:\\
- travel time spinal cord to muscle (5ms for vastus, 10ms for
gastocnemius) plus 5-10ms before muscle activation gets recognized in
surface EMG:\\
net vastus: 10-15ms\\
net gastrocnemius: 15-20ms

\subsubsection{What can be measured?}

\subsubsection{How can the operator communicate with the exoskeleton}

\begin{verbatim}
Operator uses high frequency range or patterns of output to
signal what to do
 - selection
 - parameterization (how far, how high, how fast)
\end{verbatim}

The operator controls more than the pose and motion of the
exoskeleton. What if there are "pointing" sensors or communication
devices that need to be aimed at or track an area of interest? What if
there are additional (physical or virtual) pan/tilt/zoom cameras
pointing to the side and rear or line of sight secure communications,
for example. What about other controls, such as power assist level or
thermal control? How can an operator naturally express intent and
control these additional degress of freedom?

\subsection{Operator load and fatigue}

To what extent can task recognition and "autonomy" reduce operator
load and fatigue?

\section{What are the specs?}

\subsection{What speeds and forces are needed to walk?}

\subsection{What speeds and forces are needed to run?}

\section{What performance is currently available?}

\subsection{Reference Design: linear hydraulic}

Survey state of the art using linear hydraulic actuators:
Sarcos Primus Humanoid, Atlas(BDI), Cheetah(BDI)

\subsection{Reference Design: linear electric}

Survey state of the art using linear electric actuators (typically ballscrews):
Roboray (Samsung)

\subsection{Reference Design: rotary electric}

Survey state of the art using rotary electric actuators:
Schaft~\cite{shaft_foot_placement,shaft_push_recov}, Cheetah(MIT)

\section{Sensing}

\begin{verbatim}
sensing
 actuator length/angle
 actuator velocity
 actuator force/torque 
 actuator sensing is pre-transmission
  joint backlash
  joint play
  structural flexibility
  actuator and transmission friction (10 Nm in Atlas)
 joint angle
 joint velocity
 joint torque
 IMUs all over
 cameras all over
\end{verbatim}

\section{Energy Storage}

We will only address energy storage
in terms of control issues.

\subsection{Global energy storage}

\begin{verbatim}
 battery
 gasoline/jet fuel/diesel
 chemical (explosive actuation: solid fuel rocket, liquid fuel rocket)
 pressurized gas
 other?
 energy boost systems (jump, escape response in biology)
  chemical (explosive actuation: solid fuel rocket, liquid fuel rocket)
  pressurized gas
  spring
  other?
\end{verbatim}

\subsection{Joint-level/local energy storage (on joint or synergy)}

Can we use energy storage to assist control: charging batteries,
pressurizing accumulators, using springs, ... ?

\begin{verbatim}
 accumulator (really an air spring)
 physical springs/damping
 supercapacitor/battery/electrical for local electric braking
\end{verbatim}

\section{Actuation}

Survey possible actuation systems.

What role do physical properties of actuation play in control? Highly
geared electric systems and hydraulic systems are inherently very
stiff and go where you want, but must be made compliant/backdrivable
using high performance (expensive and fragile) force control.
Pneumatics and Series Elastic Actuation (SEA) are inherently
compliant/backdrivable, which may reduce the performance
needed from the sensors and control system, but may not be able to
produce needed forces and speeds.

\subsection{Global actuation}

\begin{verbatim}
 hydraulic pump 
 gas (air) pump
\end{verbatim}

\subsection{Joint-level/synergy/local actuation}

\begin{verbatim}
 rotary electric
  motor
 linear electric
  ball screw (both motor and transmission)
 linear hydraulic
  moog thing
 rotary hydraulic
 linear pneumatic
 rotary pneumatic
 linear or rotary impulsive actuation (IC engine style actuation)
 energy boost systems (jump, escape response in biology)
  chemical (air bag style: explosive actuation: solid/liquid fuel rocket)
  pressurized gas
  spring
  other?
\end{verbatim}

\subsection{Flywheels}

Energy storage and actuation for high manueverability.

\section{Transmissions and Drive Trains}

\subsection{How can clutches and gear shifts help?}

How can we implement a gear shifting system (the crucial mechanical
system missing in current robots)? How do we switch from manipulation
while standing, walking, running, other rapid movement (dodging), and
power movements (jumping, moving through doors and walls)?

\subsection{Global Transmission}

\begin{verbatim}
 electric circuit
 hydraulic circuit
 gas (air) circuit
\end{verbatim}

\subsection{Joint-level/synergy/local transmission}

\begin{verbatim}
 rigid
  gears
   planetary
   other
  harmonic drive
  ball screw
  cosine attachment
  Sarcos Primus Humanoid knee - to reduce moment arm variation with angle
 hydraulic/pneumatic circuit
 variable
  belts
  cones
  fluid
  adjustable moment arm
   linear slider
   XY slider for actuation and ``joint'' on foot.
\end{verbatim}

\section{System issues}

Things that don't fit well elsewhere.

\begin{verbatim}
 bearing play, structural flexibility: +/- 2cm COM uncertainty
 how attached to human (exoskeleton)
 air bags
\end{verbatim}

\section{Control approaches}

What are current approaches to exoskeleton control?

How do exoskeleton control approaches compare with current and
possible approaches to mobile manipulation and humanoid robot control?

Role of phase estimation, behavioral clocks, and phase reset events
such as impacts.

Terms like impedance and admittance are used, but are often confusing,
as there are two inputs into the exoskeleton (operator-exo forces and
exo-world contact forces), and the output of the
exoskeleton can be exoskeleton actuator forces and torques, or exoskeleton
motion (position, velocity, and acceleration).

Useful paper: \url{http://summerschool.stiff-project.org/fileadmin/pdf/1804_C19.pdf}


\subsection{No control}

Rely on counterbalancing, mechanical springs, air springs, etc.
to reduce forces. 
Operator relies on mechanical backdrivability to drive
exokeleton to do task.

\subsection{Closed loop behavior selection, open loop behavior
execution}

Use joint/Cartesian position and velocity control of exoskeleton 
with pre-generated references or selected targets and automatically generated trajectories to move to targets or perform tasks.
Getting behavior selection is very important, operator may do it with
special interface (keyboard, game controller, ...).
Operator relies on mechanical backdrivability to correct exoskeleton behavior.

Recognizing/estimating phase goes here.

Phase Triggered References could go here.

Motion capture, motion libraries goes here.

\subsection{Active gravity compensation}

Similar to counterbalancing or physical gravity compensation, except done
with exoskeleton actuators to cancel exoskeleton weight (but not inertia).
A simple formulation is:
\begin{equation}
\mJ^{\tr} (m\vg)
\end{equation}
where $\mJ$ is a Jacobian matrix, $m$ is the weight of the exoskeleton,
and $\vg$ is the gravity vector. 
Operator relies on mechanical backdrivability to drive
exokeleton to do task.

Sarcos Primus Humanoid (ATR) video:\\
\url{https://www.youtube.com/watch?v=KNxxLm4sPys}

\subsection{Impedance Control}

What is impedance?
\url{https://en.wikipedia.org/wiki/Mechanical_impedance}
Stiffness, damping, and mass are components of an impedance, as
they map kinematic variables (position, velocity, and acceleration)
to forces and torques.

The goal here is to make the exoskeleton imitate a desired linear dynamic system.
In the case where there is no force sensing between the operator and the
exoskeleton, the exoskeleton controller maps from exoskeleton configuration
and velocity to desired joint/actuator/synergy forces.
In this case the actuators need to generate forces and torques rather than
positions or angles, or linear or angular velocities.

\subsection{Operator-Exoskeleton Force Control/Admittance Control}

The goal here is to use force sensing between the operator and the
exoskeleton to ultimately generate exoskeleton velocities or angular velocities,
imitating a desired linear dynamic system.
This type of control is often referred to as admittance control as the
exoskeleton maps contact forces into joint motion.
Inverse dynamics control (see below)
can be used to improve this type of control performance.

What is admittance?
\url{https://en.wikipedia.org/wiki/Admittance}
Compliance, inverse damping, inverse mass are components of an admittance,
as they map force and torques to kinematic variables (position, velocity, and
acceleration). 

\subsubsection{Get out of the way control}

The exoskeleton moves as to create as little force as possible between the
operator and the exoskeleton. The actuators in this case may be force or
position sources, but ultimately operator-exoskeleton forces are mapped
to exoskeleton velocities or angular velocities.

This is how the Sarcos WEAR exoskeleton works.

\subsubsection{Operator force feedback}

The exoskeleton moves as to provide a scale version of exoskeleton-world
contact forces, or some other (usually simple) mapping. 
The actuators in this case may be force or
position sources, but ultimately operator-exoskeleton forces are mapped
to exoskeleton velocities or angular velocities to control the
operator-exoskeleton forces to desired values.

\subsubsection{Integral admittance control}

{\it [Exoskeletons] have the implicit property
of causing a virtual modification of the dynamic response of
the human limb. We use this property of the exoskeletons
action to formulate a unified control design framework called
Integral Admittance [torque to angle] Shaping, which designs exoskeleton con-
trollers capable of producing the desired dynamic response
for the assisted limb. In this framework, a virtual increase
in the admittance of the limb is produced by coupling it
to an exoskeleton that exhibits active behavior. Specifically,
our framework shapes the magnitude profile of the integral
admittance (i.e. torque-to-angle relationship) of the coupled
human-exoskeleton system, such that the desired assistance is
achieved. This framework also ensures that the coupled stability
and passivity are guaranteed.}~\cite{Nagarajan_etal_2015}

{\it ... the impedance of the coupled human-exoskeleton
system needs to be reduced below that of the unassisted
human limb. This implies that the exoskeleton needs to
cancel its own impedance first and then compensate for
at least a part of the human limb’s impedance. Therefore,
the desired exoskeleton behavior must be that of a negative
impedance ... Consequently,
the feedback gains will all be
positive. In other words, the exoskeleton controller uses
positive feedback, and hence the exoskeleton exhibits active
behavior, which is capable of performing net positive work
on the limb. ...
However, positive feedback naturally raises the question of
stability, and so we now explain how coupled stability can be
achieved. Although the exoskeleton exhibits active behavior,
which can be potentially destabilizing, the controller can be
designed such that the coupled human-exoskeleton system
is stable and passive. ... 
}~\cite{Nagarajan_etal_2015}

{\it
... Inertia compensation is more complex ...
It can be shown that using only positive acceleration
feedback ..., the gain margin of the coupled system
reduces to the moment of inertia of the exoskeleton, which
implies that the exoskeleton controller ... can at the most
compensate for the exoskeleton’s own moment of inertia
before going unstable. This implies that the moment of
inertia of the coupled human-exoskeleton system cannot be
reduced below that of the unassisted human limb, without
compromising coupled stability. However, using low-pass
filtered acceleration feedback, it can be shown that inertia 
reduction can be achieved. ...
This work uses filtered acceleration
feedback with a second-order low-pass Butterworth filter. ...
}~\cite{Nagarajan_etal_2015}

This is a form of frequency domain virtual model control.

We are skeptical that acceleration feedback (even low pass filtered acceleration
estimated by a Kalman filter) will
work well on real systems. Impacts, shock waves, and the fact that neither
the operator's body parts or the exoskeletons parts are rigid bodies and the
joints are not well defined suggest this approach would be fragile to modeling
error and unmodeled dynamics.

\begin{verbatim}
Kalman filter:
P. Canet, “Kalman filter estimation of angular velocity and accelera-
tion: On-line implementation,” McGill University, Montr ́eal, Canada,
Tech. Rep. TR-CIM-94-15, Nov. 1994.
\end{verbatim}

\subsection{Inverse dynamics control}

The goal here is to map from desired motion (in this case desired
accelerations) to exoskeleton actuator commands (and desired operator
and world contact forces).
Inverse dynamics control is a generalization of active gravity compensation to include
inertial, Coriolis, and centripetal forces (forces due to acceleration and
velocity).
Inverse dynamics can be added to open loop behavior execution, operator force
feedback control, impedance control, and admittance control.

A key question is where does desired acceleration come from?
\begin{verbatim}
  Operator-Exoskeleton Force Control/Admittance Control  
  Receding Horizon Control of optimal trajectory of 
    idealized LIPM+flywheel model (Atkeson)
  intent prediction (see above)
  neuromuscular model (Geyer, Wang, Van de Panne)
  hybrid: Optimal control + neuromuscular model (Atkeson NSF Proposal)
    - NMM provides bias through changes to optimization criterion
    - NMM provides hard constraints
    - NMM replaces components of model-based optimization scheme
\end{verbatim}

\subsection{Virtual model control}

The goal here is to make the exoskeleton imitate a desired (usually nonlinear)
dynamic system,
For running it could be a pogo stick or trampoline, for example.

There are several possible versions of this type of control:
\begin{enumerate}
\item
Map from operator-exoskeleton contact forces and system state
to exoskeleton actuator or internal forces.
\item
Map from operator-exoskeleton contact forces and system state
to exoskeleton acceleration.
\item
Map from operator-exoskeleton contact forces and system state
to exoskeleton-world contact forces.
\item
Map from exoskeleton-world contact forces and system state to exoskeleton actuator or internal forces.
\item
Map from exoskeleton-world contact forces and system state to exoskeleton acceleration.
\item
Some combination of the above.
\end{enumerate}

\begin{verbatim}
virtual model approach: hopper, compass gait (J. Pratt)
estimate trajectory (jumping) vs. program behavior (trampoline, hopper,
   compass gait)
- J' control + behavior
- Inverse dynamics + behavior
\end{verbatim}

\subsection{Task Specific Control}

Task specific control can involve switching low level control modes, or abstracting
the behavior of the exoskeleton with a hierarchy:
\begin{enumerate}
\item
Task specific control: Generate desired motions and contact forces between
the operator and the exoskeleton, and the exoskeleton and the world.
\item
Sometimes there are intermediate levels of control.
\item
Exoskeleton control: Control the exoskeleton to generate the desired motions and
desired contact forces, with the desired impedance or admittance.
\end{enumerate}

\subsection{Survey literature and current exoskeleton systems for other approaches}

\subsection{How handle hand and other non-foot contact forces?}

Admittance control?

Map external contact forces to internal contact forces (operator-exoskeleton).

\subsection{Symbiosis and Autonomy, Handling Errors, and Superhuman Reflexes}

A later white paper will discuss
possible combinations of response to user and exoskeleton
autonomy including balance control; response to trips, slips,
stumbles, and fumbles; task assistance and guidance (guide operator to
doorknob, button, or light switch); superhuman response to external
perturbations such as projectiles and explosions; and autonomous
execution (operator/exoskeleton symbiosis with multi-tasking)?

\section{Do we need physical contact?}

Current exoskeletons use physical contact with the operator to control
the exoskeleton. Can we make an exoskeleton that does not touch or
minimally touches the operator, and in which there is no power
transfer between the operator and the exoskeleton? These types of
systems might allow an operator to behave more normally, have a
greater range of movement, be less tiring, and for non-enclosing
systems, allow the operator to move independently of the system (such
as aim and fire standard weapons or dive for cover) for additional
operator performance and safety.
\begin{enumerate}
\item
A "fat suit" or "sumo suit" could fully enclose the operator, but
use distance sensing to move with the operator.
\item
A "shield" could enclose the front of an operator (or whatever
portion of the operator is threatened) and move with the operator,
\item
A "human shield" (actually a machine shield) could move ahead of
an operator, mimicking the operators movement.
\item
An "angel" could walk, jump, or fly between an operator and potential
hostile sites, and deflect or disable projectiles.
\end{enumerate}

\section{One Time Use?}

What happens if the suit is "one time use"? For example, ablative armor,
armor that reacts to incoming projectiles before they hit,
or explosive shielding could be used. Could the armor weight be decreased?
The expensive parts of the suit could be re-usable, and even autonomously
return to safe areas.

\section{Conclusions and Recommendations}

Conclusions and Recommendations to be written.

\bibliographystyle{plain}
\bibliography{exo}

\newpage

\section{Appendix: Our List of Exoskeletons}

Please tell us what is missing.

\subsection{Surveys}

\begin{verbatim}
Surveys:
http://exoskeletonreport.com/
http://robohub.org/tag/exoskeleton/
http://robohub.org/exoskeletons-new-and-older/
http://neurogadget.com/tag/exoskeleton
http://www.engadget.com/tag/exoskeleton/
http://spectrum.ieee.org/robotics/medical-robots/exoskeletons-around-the-world
http://spectrum.ieee.org/robotics/medical-robots/exoskeletons-are-on-the-march
http://spectrum.ieee.org/biomedical/bionics/the-rise-of-the-body-bots
http://www.extremetech.com/electronics/139633-will-we-ever-have-iron-man-exoskeletons
http://futurewarstories.blogspot.com/2014/08/fws-topics-combat-exoskeletons.html
https://prezi.com/jygcvfxiesnm/a-brief-introduction-to-biomechanical-exoskeletons/
http://news.discovery.com/tech/robotics/exoskeleton-robots-top-5.htm
http://blog.equipoisinc.com/new-technologies-emerging-aerospace-defense-manufacturing/
http://nextbigfuture.com/2015/03/lower-body-exoskeleton-audi-chairless.html

A. M. Dollar, H. Herr, Lower extremity exoskeletons and active
orthoses: challenges and state-of-the-art, IEEE
Transactions on Robotics: 24 (2008) 144-158.
(surveys exoskeletons)
http://excedrin.media.mit.edu/wp-content/uploads/sites/3/2013/07/Dollar-2008_Lower-Extremity-Exoskeletons-and-Active-Orthoses-Challenges-and-State-of-the-Art.pdf
\end{verbatim}

\subsection{Exoskeletons}

\begin{verbatim}
BLEEX: Berkeley Lower Extremity Exoskeleton

Ekso Bionics
http://www.bloomberg.com/bw/articles/2013-03-28/businesses-bet-on-iron-man-like-exoskeletons

Sarcos WEAR
http://www.army-technology.com/features/featuremilitary-exoskeletons-uncovered-ironman-suits-a-concrete-possibility
http://www.robaid.com/bionics/raytheons-second-generation-exoskeleton-xos-2.htm

Lockheed Martin HULC, FORTIS
http://www.lockheedmartin.com/us/news/features/2014/mfc-103114-relief-daily-grind-industrial-exoskeletons-work.html
http://www.lockheedmartin.com/us/products/exoskeleton/FORTIS.html
http://www.dailytech.com/From+HULC+to+FORTIS+the+Evolution+of+Lockheed+Martins+Incredible+Exosuit/article36421.htm
http://www.wired.com/2014/09/navys-exoskeleton-could-make-workers-20-times-more-productive/
http://robrady.com/design-project/lockheed-martin-fortis-human-powered-exoskeleton
http://www.cnn.com/2013/05/22/tech/innovation/exoskeleton-robot-suit/

DKFI CAPIO and Exoskeleton Active (VI-Bot)
http://robotik.dfki-bremen.de/en/research/robot-systems/exoskeleton-passive.html
http://robotik.dfki-bremen.de/en/research/robot-systems/exoskeleton-active.html

DAEWOO
http://mashable.com/2014/08/05/daewoo-exoskeleton/#xz8IUKO3bmk_
http://www.geek.com/science/daewoo-dock-workers-now-using-strength-enhancing-exo-skeletons-1601290/

Vanderbilt
http://www.dailytech.com/Iron+Man+New+Robotic+Exoskeleton+Helps+Paraplegics+Walk/article29079.htm
http://research.vuse.vanderbilt.edu/cim/research_orthosis.html

Vanderbilt/Parker-Hannafin
http://www.indego.com/indego/en/home
https://www.youtube.com/watch?v=-bYYmZNxaZk
https://en.wikipedia.org/wiki/Vanderbilt_exoskeleton
http://www.roboticstrends.com/article/parker_hannifin_indego_exoskeleton
http://www.crainscleveland.com/article/20141112/BLOGS03/141119915/parker-hannifin-accepts-the-challenge-of-commercializing-the-indego--41951672-812546791-1442419256=:6352--

Activelink/Panasonic
http://pinktentacle.com/2009/09/power-loader-exoskeleton-suit/
http://www.extremetech.com/extreme/173463-worlds-first-affordable-powered-exoskeleton-is-almost-here-prepare-for-mech-wars
http://www.dealstreetasia.com/stories/panasonic-to-release-commercial-exoskeleton-to-market-in-4q-2015-9269/

Cyberdyne: HAL 2000 up to at least HAL 5
http://www.computerweekly.com/photostory/2240108388/Photos-Cyberdyne-Hal-robotic-exoskeleton-to-help-paralyzed/7/Japan-beats-the-US-to-it-Cyberdyne-Hal-robotic-exoskeleton-to-help-paralyzed
http://spectrum.ieee.org/robotics/medical-robots/exoskeletons-are-on-the-march
http://www.dailymail.co.uk/sciencetech/article-2384930/Robotic-exoskeleton-help-rehabilitate-disabled-people-passes-safety-tests--paving-way-sale-UK.html

Noonee
http://www.extremetech.com/extreme/188417-rejoice-commuters-and-workers-the-chairless-chair-exoskeleton-lets-you-sit-down-anywhere-anytime

Kanagawa Institute of Technology
pneumatic exoskeleton developed at the Kanagawa Institute of Technology,
in Atsugi, Japan, allows Akiko Michihisa, a fitness trainer, ...
http://www.ubergizmo.com/2007/10/air-pressure-exoskeleton/

Monash/Chen
http://www.dailymail.co.uk/sciencetech/article-2639939/Super-fireman-The-hi-tech-exoskeleton-firefighters-superhuman-abilities.html
http://news.discovery.com/tech/robotics/firefighter-exoskeleton-to-the-rescue-140521.htm

Sagawa Electronics
https://www.youtube.com/watch?v=tYgHdxmCcAI
http://www.ohgizmo.com/2013/07/11/mk3-exoskeleton-suit-promises-to-make-schoolgirls-better-taller-stronger/
http://www.bitrebels.com/technology/powered-jacket-exoskeleton-japan/

Micromega
http://www.micromega-dynamics.com/esa-projects/exoskeleton.html

UCLA
http://bionics.seas.ucla.edu/research/exoskeleton_device_3.html
http://bionics.seas.ucla.edu/research/exoskeleton_device_2.html

NASA/IHMC
http://www.nasa.gov/offices/oct/home/feature_exoskeleton.html#.VfHU72wVhBc
http://www.dailymail.co.uk/sciencetech/article-2216748/Nasa-developing-exoskeleton-help-astronauts-exercise-zero-gravity-help-disabled-people-walk-Earth.html
http://robots.ihmc.us/x1-mina-exoskeleton/

DARPA Warrior Web program
http://gajitz.com/darpas-exoskeleton-can-help-soldiers-run-4-minute-miles/
http://www.dailymail.co.uk/sciencetech/article-2348174/Robotic-exoskeleton--soft-light-pulls-like-pair-trousers--day-soldiers-superhuman-strength.html
http://www.theengineer.co.uk/in-depth/the-big-story/power-dressing-why-its-exoskeleton-time/1019633.articlehttp://www.theengineer.co.uk/in-depth/the-big-story/power-dressing-why-its-exoskeleton-time/1019633.article
http://www.engadget.com/2014/09/12/darpa-harvard-soft-exosuit/

ESA
http://www.esa.int/spaceinimages/Images/2014/06/Ready_to_wear_exoskeleton
http://www.esa.int/spaceinimages/Images/2014/06/Exoskeleton
http://www.sciencefocus.com/blog/exoskeleton-turning-man-machine

3D Systems
http://www.dezeen.com/2014/03/05/3d-printed-exoskeleton-helps-paralysed-users-walk/

U.Penn
http://www.popsci.com/article/technology/invention-awards-2014-powerful-portable-and-affordable-robotic-exoskeleton
https://www.youtube.com/watch?v=kAyI1YrWThQ

MIT Herr:
http://biomech.media.mit.edu/portfolio_page/load-bearing-exoskeleton-for-augmentation-of-human-running/

MIT exo
http://biomech.media.mit.edu/portfolio_page/load-bearing-exoskeleton-for-walking/
http://www.technovelgy.com/ct/Science-Fiction-News.asp?NewsNum=1205

Honda (Stride Management Assist (SMA))
http://www.slashgear.com/honda-demo-walking-assist-exoskeleton-for-elderly-disabled-2211295/
http://singularityhub.com/2010/09/13/hondas-exoskeletons-help-you-walk-like-asimo-video/

REX exoskeleton
http://www.gizmag.com/rex-robotic-exoskeleton/15736/pictures
http://www.stripes.com/news/exoskeleton-could-benefit-troops-with-spinal-cord-injuries-1.113657
http://www.funis2cool.com/cool/rex-the-robotic-exoskeleton-could-change-the-lives-of-disabled-people.html

Waterloo
http://www.eng.uwaterloo.ca/~jpiorkow/fydp/

Soft pneumatic exo
http://cwwang.com/2008/04/08/soft-pneumatic-exoskeleton/

ReWalk
http://www.businessinsider.com/human-exoskeleton-approved-by-fda-2014-6

Revision Military PROWLER Human Augmentation System (HAS) Exoskeleton
https://www.youtube.com/watch?v=Zq4amM9u-6o

Yasakawa
http://www.crunchwear.com/yaskawa-electrics-ankle-exoskeleton-gives-strength/

http://www.engadget.com/2015/08/20/japans-top-oil-company-is-building-an-aliens-power-loader/

http://robohub.org/clinical-trials-begin-for-russias-first-medical-exoskeleton/

https://en.wikipedia.org/wiki/Z_series_(space_suits)

https://en.wikipedia.org/wiki/Extravehicular_Mobility_Unit

Passive:
http://techcrunch.com/2011/03/23/japanese-engineers-create-human-powered-exoskeleton-suit-video/
http://www.jneuroengrehab.com/content/6/1/21/figure/F2?highres=y

SUEFUL-7: A 7DOF upper-limb exoskeleton robot with muscle-model-oriented
EMG-based control
RARC Gopura, K Kiguchi, Y Li - Intelligent Robots and Systems …, 2009 -
ieeexplore.ieee.org

EU BALANCE program
http://balance-fp7.eu/
points us to European work
http://cordis.europa.eu/project/rcn/106854_en.html

MINDWALKER
http://www.3me.tudelft.nl/en/about-the-faculty/departments/biomechanical-engineering/research/dbl-delft-biorobotics-lab/exoskeleton/

https://mindwalker-project.eu/

http://www.utwente.nl/ctw/bw/research/projects/MINDWALKER/
http://tnsre.embs.org/2015/03/27/design-control-mindwalker-exoskeleton/

HEIKA
https://www.heika-research.de/26_166.php
HEiKA-EXO - Optimization-based development and control of an exoskeleton
for medical applications

http://www.beyondhd.tv/exoskeleton/

https://www.youtube.com/watch?t=3&v=myNRcDIgfjw
www.livescience.com/51940-mind-controlled-exoskeleton-robot.html
http://www.ibtimes.co.uk/scientists-develop-mind-controlled-robotic-exoskeleton-that-uses-leds-help-paraplegics-walk-1516834

http://csl-ep.mech.ntua.gr/index.php/research/telerobotics-and-servohydraulic-simulators/emg-exoskeleton-control

http://www.sarna.net/news/maxfas-exoskeleton-human-fire-control-supports-against-fatigue/
MAXFAS Exoskeleton: Human Fire Control Supports against Fatigue

Historical
Cornell Aeronautical Labs with funding from the US Air Force. This
resulted in the 1961-1962 Man-Amplifier suit that was more of an working
model than a working prototype.
The Cornell exo-suit was a mockup, and never included motors, servos, or
power.

GE's Hardiman I industrial exoskeleton project from 1965-1971 was the
first working exo-suit.
While a full mockup model was constructed, only pieces of Hardiman I
exo-suit worked, namely the claw-like arms. While the arms worked, the
leg-based frame work were an issue, especially with locomotion. By 1971,
GE canceled the project.
http://www.neatorama.com/2008/05/04/building-your-own-functional-iron-man-suit/

the best of the 1980's exo-suits was the 1988 NASA AX-5 Hardshell space
suit. For several decades, NASA's AMES had been working on a hadshelled
space suit system that they called the "AX" series. The apex of this
research was the 1988 AX-5.
While bulky, the AX-5 possessed 95% maneuverable of the naked human form,
and this suit was targeted for protection of astronauts in high-risk
areas. It was envisioned that the AX-5 would be used in EVAs in Earth
orbit and asteroid mining. In the end, it was determined that NASA did not
need a suit like this and alone with some remaining technical issues, the
AX hardsuits were shelved.
\end{verbatim}

\newpage

\section{Appendix: References To Be Processed}

Please suggest additional references.

\begin{verbatim}

Overviews ********************************

Active Exoskeleton Control Systems: State of the Art
http://www.sciencedirect.com/science/article/pii/S1877705812026732

Control methodologies for upper limb exoskeleton robots
http://ieeexplore.ieee.org/xpl/login.jsp?tp=&arnumber=6427387&url=http%3A%2F%2Fieeexplore.ieee.org%2Fiel5%2F6416132%2F6426928%2F06427387.pdf%3Farnumber%3D6427387

A review of exoskeleton-type systems and their key technologies
CJ Yang, JF Zhang, Y Chen, YM Dong
http://pic.sagepub.com/content/222/8/1599.abstract

D. P. Ferris, “The exoskeletons are here,” Journal of Neuroengineering
and Rehabilitation, vol. 6, no. 17, pp. 1–3, 2009.

G. S. Sawicki and D. P. Ferris, “Mechanics and energetics of level
walking with powered ankle exoskeletons,” J. Exp. Biol., vol. 211, no.
Pt. 9, pp. 1402–1413, 2008.

http://www.bme.ncsu.edu/labs/hpl/SawickiDissertationFinal.pdf
Mechanics and energetics of walking with powered exoskeletons

Carry Payload *****************************

The first autonomously powered exoskeleton device that experimentally
demonstrated reduction in metabolic cost for human walking during
load carriage
L. M. Mooney, E. J. Rouse, and H. M. Herr, “Autonomous exoskeleton
reduces metabolic cost of human walking during load carriage,”
Journal of Neuroengineering and Rehabilitation, vol. 11, no. 80, 2014.

H. Kazerooni and R. Steger, “Berkeley lower extremity exoskeleton,”
ASME J. Dyn. Syst., Meas., Control, vol. 128, pp. 14–25, 2006.

C. J. Walsh, K. Pasch, and H. Herr, “An autonomous, underactuated
exoskeleton for load-carrying augmentation,” in Proc. IEEE/RSJ Int.
Conf. Intelligent Robots and Systems (IROS), 2006, pp. 4110–1415.

Impedance control **************************

G. A.-Ollinger, J. E. Colgate, M. A. Peshkin, and A. Goswami,
“Design of an active one-degree-of-freedom lower-limb exoskeleton
with inertia compensation,” Int. J. Robotics Research, vol. 30, no. 4,
pp. 486–499, 2011.

Learning ************************************

K. E. Gordon, C. R. Kinnaird, and D. P. Ferris, “Locomotor adaptation
to a soleus emg-controlled antagonistic exoskeleton,” J. Neurophysiol.,
vol. 109, no. 7, pp. 1804–1814, 2013.

EMG control ********************************

H. Kawamoto, S. Lee, S. Kanbe, and Y. Sankai, “Power assist method
for hal-3 using emg-based feedback controller,” in Proc. IEEE Int.
Conf. Syst., Man, Cybern., 2003, pp. 1648–1653.

S. Lee and Y. Sankai, “The natural frequency-based power assist
control for lower body with hal-3,” in Proc. IEEE Int. Conf. Sys. Man
Cyber. (ICSMC), 2003, pp. 1642–1647.

Not sorted ***********************************

master thesis design and control of a robotic exoskeleton for ...
digital.csic.es/.../Bortole_M_Design_Control_Robotic_Exoskeleton_Gait...
by M Bortole - ‎2013 - ‎Cited by 1
Control of a Robotic Exoskeleton for Gait Rehabilitation ” written by
Magdo Bortole as a requirement to obtain the Degree in the official Master
in Robotics and

Active Impedance:
A Novel Concept in Assistive Exoskeleton Control
http://www.ambarish.com/exos.html

http://gradworks.umi.com/33/86/3386362.html
Active impedance control of a lower-limb assistive exoskeleton
by Aguirre-Ollinger, Gabriel, Ph.D., NORTHWESTERN UNIVERSITY, 2009, 297
pages; 3386362

Active-impedance control of a lower-limb assistive exoskeleton
G Aguirre-Ollinger, JE Colgate… - … , 2007. ICORR 2007. …, 2007 -
ieeexplore.ieee.org

Design of an active one-degree-of-freedom lower-limb exoskeleton with
inertia compensation
G Aguirre-Ollinger, JE Colgate… - … Journal of Robotics …, 2011 -
ijr.sagepub.com

http://www.scholars.northwestern.edu/pubDetail.asp?t=pm&id=84856481679&
Inertia compensation control of a one-degree-of-freedom exoskeleton for
lower-limb assistance: Initial experiments

Power assist control for walking aid with HAL-3 based on EMG and impedance
adjustment around knee joint
S Lee, Y Sankai - Intelligent Robots and Systems, 2002. IEEE/ …, 2002 -
ieeexplore.ieee.org

http://www.intechopen.com/books/international_journal_of_advanced_robotic_syste\
ms/design-and-control-of-a-powered-hip-exoskeleton-for-walking-assistance
Design and Control of a Powered Hip Exoskeleton for Walking Assistance
By Qingcong Wu, Xingsong Wang, Fengpo Du and Xiaobo Zhang

http://www.jcomputers.us/vol9/jcp0901-29.pdf
Carrying Lower Extreme Exoskeleton Rapid
Terminal Sliding-Mode Robust Control

http://h2t.anthropomatik.kit.edu/pdf/Beil2015.pdf
Design and Control of the Lower Limb Exoskeleton KIT-EXO-1

Reducing the energy cost of human walking using an unpowered exoskeleton
http://www.nature.com/nature/journal/v522/n7555/fig_tab/nature14288_SF1.html

Control of a lower extremity exoskeleton for human performance
amplification
JLC Racine - 2003 - dl.acm.org

The RoboKnee: an exoskeleton for enhancing strength and endurance during
walking
JE Pratt, BT Krupp, CJ Morse… - … and Automation, 2004. …, 2004 -
ieeexplore.ieee.org

Locomotive control of a wearable lower exoskeleton for walking enhancement
KH Low, X Liu, CH Goh, H Yu - Journal of Vibration and Control, 2006 -
jvc.sagepub.com

Exoskeleton robots for upper-limb rehabilitation: state of the art and
future prospects
HS Lo, SQ Xie - Medical engineering & physics, 2012 - Elsevier

http://www.scielo.org.co/scielo.php?pid=S0123-30332014000100020&script=sci_arttext
Design and Implementation of a control strategy for static balance of a
lower limbs exoskeleton

Implementation of Real Time Control Algorithm for Gait ...
www.seipub.org/ABER/Download.aspx?ID=22038
Control approach follows gait trajectory using feedback sensors and
actuators for movement control. ... Exoskeleton Device; Real Time Control;
Gait Phases.

Exoskeleton control for lower-extremity assistance based on adaptive
frequency oscillators: Adaptation of muscle activation and movement
frequency
http://pih.sagepub.com/content/229/1/52.abstract

Control of a Lower Extremity Exoskeleton for Human ...
202.28.199.34/multim/3121656.pdf
The lower extremity exoskeleton is a wearable robotic device that should
... The control law makes use of dynamic models of the exoskeleton to
estimate the.

http://labrob.mty.itesm.mx/proyectos/robotic-exoskeletons
Robotic Exoskeletons

http://info.scichina.com/sciFe/EN/abstract/abstract516165.shtml
The relationship between physical human-exoskeleton interaction and
dynamic factors: using a learning approach for control applications

Online Assessment of Human-Robot Interaction for Hybrid Control of Walking
http://www.mdpi.com/1424-8220/12/1/215


\end{verbatim}

\end{document}


