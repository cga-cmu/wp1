\section{What are some design philosophy alternatives?}

\subsection{``Invisible'' exoskeleton}

Can we build an exoskeleton allows the operator to behave naturally
and exerts little or no force on the operator?

This design goal
could be achieved through active control using feedback of the
operator-exoskeleton forces and/or high quality prediction of what
the operator will do next. Additional gravity, friction, actuator, and
inertial compensation (inverse dynamics) can improve performance.

\subsection{``Natural'' exoskeleton}

Can we build an exoskeleton allows the operator to behave naturally?
The operator can feel the exoskeleton, but can overpower it when necessary.

One way to achieve this design goal is to have extremely light attachments
to the limbs, and put all heavy exoskeleton components on the
torso and head. The limbs can be physically driven by the operators muscles
with little ``drag'' on the operator.
Bypass valves and clutches could allow the operator to physically ``take over''
and push the relatively lightweight limb mounted parts of the exoskeleton around.

A complementary way to achieve this goal is through active control. Low impedance
(torque source) actuation makes combining active control with operator backdriving
easier.

A ``natural'' exoskeleton is easier to build than 
an ``invisible'' exoskeleton.

\subsection{``Symbiotic'' exoskeleton}

Can we build an exoskeleton that allows the operator/exoskeleton team
to be effective, but not necessarily behave naturally? 
The operator needs to learn to ``fly'' the suit,
just as operators learn to operate parachutes,
wing suits, diving equipment, high and low temperature protective
suits, firefighting equipment, and high altitude flight
and parachute suits.

An advantage of this approach is that 
the exoskeleton can adapt its behavior to the task.
For example, for jumping, the exoskeleton can operate like a pogo 
stick:\\
\url{https://www.youtube.com/watch?v=lbp41vWP4o4},\\ 
and for running it could act like jumping stilts:\\
\url{https://www.youtube.com/watch?v=9ZOd7yEyhwI},

A ``symbiotic'' exoskeleton is easier to build than an ``invisible'' or ``natural''
exoskeleton.
